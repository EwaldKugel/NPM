\section{Chancen und Risiken von Smart Government}
Die Verwaltungsmodernisierung hat das Ziel, die Effizienz, Wirksamkeit und Bürgernähe der Verwaltung zu verbessern.
Ein Ansatz zur Verwaltungsmodernisierung ist das Smart Government, das die Verwaltung durch den Einsatz von Informationstechnologie und digitalen Lösungen verbessern soll.
Beim Smart Government bewegen sich viele Anwendungsfelder in der öffentlichen Verwaltung auf einem schmalen Grat zwischen den Chancen, die Smart Government mitbringt und Risiken, wie Datenschutz, Überwachung und Fremdsteuerung \citep[Vgl.][]{Lucke2018}.
Von Lucke beschreibt vor allem den ethischen Aspekt bei der Nutzung von neuen Technologien, so sollte Smart Government nach den eigenen ethischen Vorstellungen zusammen mit der Bevölkerung gestaltet werden \citep[Vgl.][S. 108]{Lucke2018}.
\par
Dabei sollten die Handlungsfelder, in denen autonome Technologien im Rahmen des Smart Governments eingesetzt werden, stark überdacht werden.
So ist in den meisten Fällen davon abzuraten autonome Computersysteme bei wichtigen Entscheidungen, zum Beispiel der Entscheidung über Haftstrafen, zu benutzen.
Laut Tonn und Stiefel könnte es sogar einen Verlust an Menschlichkeit geben, sollten solche Entscheidungen nicht mehr durch den Menschen getroffen werden \citep[Vgl.][S. 322]{Tonn2012}.
\par 
\subsection{Verbesserung der Effizienz innerhalb der Verwaltung}
Ein Ansatz zur Verwaltungsmodernisierung ist das Smart Government, das die Verwaltung durch den Einsatz von Informationstechnologie und digitaler Lösungen verbessern soll.
Beim Smart Government bewegen sich viele Anwendungsfelder in der öffentlichen Verwaltung auf einem schmalen Grat zwischen den Chancen, die Smart Government mitbringt und Risiken, wie Datenschutz, Überwachung und Fremdsteuerung \citep[Vgl.][]{Lucke2018}.
\par
Ein Ziel, das beim Ansatz des Smart Governments zu betrachten ist, ist die Verbesserung der Effizienz.
Durch den Einsatz von IT-Lösungen, Datenanalyse und automatisierten Verfahren kann Smart Government die Verwaltungsprozesse beschleunigen und standardisieren. 
Als grobes Oberziel ist es daran, die papierbasierten Abläufe durch elektronische Akten- und Vorgangsbearbeitungssysteme abzulösen, um die Effizienz zu steigern \citep[][]{von_Lucke_2016}.
\par
So liegt laut von Lucke das größte Potential zur Verbesserung der Effizienz darin, nicht die Eins-zu-eins-Umsetzung von papierbasierten Abläufen in intelligentes Papier, sondern die komplette Überarbeitung dieser Vorgänge in rein digitale Verwaltungsprozesse zu erneuern \citep[][S.179]{von_Lucke_2016}.
Das Auslagern dieser Prozesse in solche Akten- und Vorgangsbearbeitungssysteme ermöglicht eine stärkere und schnelle Massenbearbeitung von Anträgen, Rechnungen und Genehmigungsprozessen, was eine deutliche Erhöhung der Effizienz bedeutet.
\par
Ein Beispiel, indem diese Effizienzsteigerung durch automatische Vernetzung von Daten in Verwaltungsakte umgesetzt wurde kommt dabei aus Schweden.
Schweden schuf mit Hilfe der Online-Plattform SSBTEK eine Verknüpfung von Daten aus nationalen Registern (z.B. Steuerbehörde, Renten- und Sozialversicherung, Arbeitsagentur, Studienförderung) um diese direkt in den Antrag auf Sozialhilfe einfließen zu lassen \citep[][]{bitkom2023}.
Damit umgeht Schweden den vorherigen langwierigen Verwaltungsprozess, die Daten über mehrere Verwaltungsschritte anzufragen und zu erhalten.
Somit lassen sich mit Hilfe von wenigen Klicks die benötigten Daten Online in den Sozialhilfeantrag einfügen, und diesen an das zuständige Amt abschicken \citep[][]{NationalenNormenkontrollrat2017}.
\par
Natürlich gibt es bei der Umsetzung zur Digitalisierung auch einige Risiken, wie der korrekten Behandlung von Daten bezüglich des Datenschutzes.
Allerdings liegt der Fokus ganz klar auf der Gestaltung einer effizienteren Verwaltung, da die Effizienz hier dir Risiken überwiegen.
Durch den Einsatz von IT-Lösungen und automatisierten Verfahren kann die Verwaltung also deutlich effizienter, besonders in der Bearbeitung von Bürgeranträgen, gestaltet werden.
\input{VorteileVonSmartGovernment/ErhöhungDerTransparenzUnterSmartGovernment}
\subsection{Verstärkung der Bürgerbeteiligung an der Verwaltung}\label{VortBürgerbeteiligung}
Bürgerbeteiligung ist ein wesentlicher Bestandteil einer funktionierenden Demokratie. 
Sie ermöglicht es den Bürgern, Einfluss auf die Entscheidungen der Verwaltung zu nehmen, indem sie ihre Meinung kundtun und an der Entscheidungsfindung teilnehmen.
Gleichzeitig wird die Relevanz der Verwaltung erhöht, da es den Bürgern ermöglicht, ihre spezifischen Bedürfnisse anzusprechen und sie in die Entscheidungsfindung einzubeziehen.
Laut Schuppan wird die Qualität von Verwaltungen durch verbesserte Leistungsangebote, etwa durch digitalisierte Verwaltungsprozesse, die sich stärker mit dem Bürger beschäftigen, verbessert \citep[Vgl.][S.35]{Schuppan2016}.
Damit ist eine hohe Bürgerbeteiligung ein hohes Qualitätskriterium bei der Gestaltung einer Verwaltungsmodernisierung.
Folgend werden einige Projekte und Ansätze zur Erhöhung der Bürgerbeteiligung im Zuge des Smart Governments diskutiert.
\par
Generell leitet sich die Bürgerbeteiligung an der Verwaltung per Onlineangeboten an den Grundprinzipien des Open Government, also der Öffnung von Verwaltungsprozessen durch Behörden, ab.
Eine wichtige Säule dieser Open Government Bewegung ist die Partizipation von Bürgern an der Verwaltung oder Politik (siehe Abbildung \ref{fig:DreiSäulenOpenGovernment}).
\begin{figure}[h]
 \centering
 \includegraphics{Assets/DreiSäulenOpenGovernment.png}
 \caption{Säulen des Open Governments \citep[][]{Leitner2018}.}
 \label{fig:DreiSäulenOpenGovernment}
\end{figure}
Beispiele für diese E-Partizipation an der Regierung sind digitale Volksabstimmungen oder Volksbefragungen.
Zwar gibt es Hürden wie die Identifikation der Bürger bei solchen E-Partizipationen, allerdings fällt die Abwegung von Chance durch Partizipation und das Risiko der Identifikation so aus, dass die Chance dem Risiko überwiegt \citep[][S. 15]{Leitner2018}.
\par
Allgemein konzentriert sich Smart Government in Bezug auf die Bürgerbeteiligung darauf, Bürgern die Möglichkeit zu geben, sich aktiv an der Gestaltung ihrer Kommunen zu beteiligen. 
Es geht darum, die Fähigkeiten und Ideen der Bürger zu nutzen, um eine bessere Zusammenarbeit zwischen Bürgern und Verwaltung zu fördern.  
Eine Reihe von Initiativen und Projekten in Deutschland setzen sich für eine stärkere Bürgerbeteiligung ein, um mehr Mitbestimmung und Partizipation zu fördern \citep[][S. 18, 49-50]{Leitner2018}.
\par 
Diese Beteiligungen können beispielweise digitale Bürgersprechstunden in Form von Videochats oder die Kommunikation mit der Verwaltung über Social-Media-Plattformen sein.
Wichtig dabei ist, dass die Bürger Feedback erhalten und bemerken dass ihre Meinungen und Anregungen angehört werden.
\par
So wird durch Smart Government den Bürgern die Chance geboten sich aktiv am politischen Prozess und an der Entscheidungsfindung zu beteiligen. 
Dies kann durch neue Technologien umgesetzt und verbessert in der zukünftigen Verwaltung angewandt werden.
Dadurch wird die Transparenz in der Verwaltung gestärkt und die Bürger können ihre Interessen aktiv vertreten, was den Grundgedanken der Demokratie leben lässt und verstärkt. 
