\subsection{Verstärkung der Bürgerbeteiligung an der Verwaltung}\label{VortBürgerbeteiligung}
Bürgerbeteiligung ist ein wesentlicher Bestandteil einer funktionierenden Demokratie. 
Sie ermöglicht es den Bürgern, Einfluss auf die Entscheidungen der Verwaltung zu nehmen, indem sie ihre Meinung kundtun und an der Entscheidungsfindung teilnehmen.
Gleichzeitig wird die Relevanz der Verwaltung erhöht, da es den Bürgern ermöglicht, ihre spezifischen Bedürfnisse anzusprechen und sie in die Entscheidungsfindung einzubeziehen.
Laut Schuppan wird die Qualität von Verwaltungen durch verbesserte Leistungsangebote, etwa durch digitalisierte Verwaltungsprozesse, die sich stärker mit dem Bürger beschäftigen, verbessert \citep[Vgl.][S.35]{Schuppan2016}.
Damit ist eine hohe Bürgerbeteiligung ein hohes Qualitätskriterium bei der Gestaltung einer Verwaltungsmodernisierung.
Folgend werden einige Projekte und Ansätze zur Erhöhung der Bürgerbeteiligung im Zuge des Smart Governments diskutiert.
\par
Generell leitet sich die Bürgerbeteiligung an der Verwaltung per Onlineangeboten an den Grundprinzipien des Open Government, also der Öffnung von Verwaltungsprozessen durch Behörden, ab.
Eine wichtige Säule dieser Open Government Bewegung ist die Partizipation von Bürgern an der Verwaltung oder Politik (siehe Abbildung \ref{fig:DreiSäulenOpenGovernment}).
\begin{figure}[h]
 \centering
 \includegraphics{Assets/DreiSäulenOpenGovernment.png}
 \caption{Säulen des Open Governments \citep[][]{Leitner2018}.}
 \label{fig:DreiSäulenOpenGovernment}
\end{figure}
Beispiele für diese E-Partizipation an der Regierung sind digitale Volksabstimmungen oder Volksbefragungen.
Zwar gibt es Hürden wie die Identifikation der Bürger bei solchen E-Partizipationen, allerdings fällt die Abwegung von Chance durch Partizipation und das Risiko der Identifikation so aus, dass die Chance dem Risiko überwiegt \citep[][S. 15]{Leitner2018}.
\par
Allgemein konzentriert sich Smart Government in Bezug auf die Bürgerbeteiligung darauf, Bürgern die Möglichkeit zu geben, sich aktiv an der Gestaltung ihrer Kommunen zu beteiligen. 
Es geht darum, die Fähigkeiten und Ideen der Bürger zu nutzen, um eine bessere Zusammenarbeit zwischen Bürgern und Verwaltung zu fördern.  
Eine Reihe von Initiativen und Projekten in Deutschland setzen sich für eine stärkere Bürgerbeteiligung ein, um mehr Mitbestimmung und Partizipation zu fördern \citep[][S. 18, 49-50]{Leitner2018}.
\par 
Diese Beteiligungen können beispielweise digitale Bürgersprechstunden in Form von Videochats oder die Kommunikation mit der Verwaltung über Social-Media-Plattformen sein.
Wichtig dabei ist, dass die Bürger Feedback erhalten und bemerken dass ihre Meinungen und Anregungen angehört werden.
\par
So wird durch Smart Government den Bürgern die Chance geboten sich aktiv am politischen Prozess und an der Entscheidungsfindung zu beteiligen. 
Dies kann durch neue Technologien umgesetzt und verbessert in der zukünftigen Verwaltung angewandt werden.
Dadurch wird die Transparenz in der Verwaltung gestärkt und die Bürger können ihre Interessen aktiv vertreten, was den Grundgedanken der Demokratie leben lässt und verstärkt.