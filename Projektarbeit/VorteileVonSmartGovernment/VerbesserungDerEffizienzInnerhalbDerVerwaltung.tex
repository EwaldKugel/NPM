\subsection{Verbesserung der Effizienz innerhalb der Verwaltung}
Ein Ansatz zur Verwaltungsmodernisierung ist das Smart Government, das die Verwaltung durch den Einsatz von Informationstechnologie und digitalen Lösungen verbessern soll.
Beim Smart Government bewegen sich viele Anwendungsfelder in der öffentlichen Verwaltung auf einem schmalen Grat zwischen den Chancen, die Smart Government mitbringt und Risiken, wie Datenschutz, Überwachung und Fremdsteuerung \citep[Vgl.][]{Lucke2018}.
\par
Ein Ziel, das beim Ansatz des Smart Governments zu betrachten ist, ist die Verbesserung der Effizienz.
Durch den Einsatz von IT-Lösungen, Datenanalyse und automatisierten Verfahren kann Smart Government die Verwaltungsprozesse beschleunigen und standardisieren. 
Als grobes Oberziel ist es daran, die papierbasierten Abläufe durch elektronische Akten- und Vorgangsbearbeitungssysteme abzulösen, um die Effizienz zu steigern \citep[][]{von_Lucke_2016}.
\par
So liegt laut von Lucke das größte Potential zur Verbesserung der Effizienz darin, nicht die Eins-zu-eins-Umsetzung von papierbasierten Abläufen in intelligentes Papier, sondern die komplette Überarbeitung dieser Vorgänge in rein digitale Verwaltungsprozesse \citep[][S.179]{von_Lucke_2016}.
Das Auslagern dieser Prozesse in das solche Akten- und Vorgangsbearbeitungssysteme ermöglicht eine stärkere und schnelle Massenbearbeitung von Anträgen, Rechnungen und Genehmigungsprozessen, was eine deutliche Erhöhung der Effizienz bedeutet.
\par
Ein Beispiel, indem diese Effizienzsteigerung durch automatische Vernetzung von Daten in Verwaltungsakte umgesetzt wurde kommt dabei aus Schweden.
Schweden schuf mit Hilfe der Online-Plattform SSBTEK eine Verknüpfung von Daten aus nationalen Registern (z.B. Steuerbehörde, Renten- und Sozialversicherung, Arbeitsagentur, Studienförderung) um diese direkt in den Antrag auf Sozialhilfe einfließen zu lassen.
Damit umgeht Schweden den vorherigen langwirigen Verwaltungsprozess, die Daten über mehrere Verwaltungsschritte anzufragen und zu erhalten.
Somit lassen sich mit Hilfe von wenigen Klicks die benötigten Daten Online in den Sozialhilfeantrag einfügen, und diesen an das zuständige Amt abschicken \citep[][]{NationalenNormenkontrollrat2017}.
\par
Natürlich gibt es bei der Umsetzung zur Digitalisierung auch einige Risiken, wie der korrekten Behandlung von Daten bezüglich des Datenschutzes.
Allerdings liegt der Fokus ganz klar auf der Gestaltung einer effizienteren Verwaltung, da die Effizienz hier dir Risiken überwiegen.
Durch den Einsatz von IT-Lösungen und automatisierten Verfahren kann die Verwaltung also deutlich effizienter, besonders in der Bearbeitung von Bürgeranträgen, gestaltet werden.