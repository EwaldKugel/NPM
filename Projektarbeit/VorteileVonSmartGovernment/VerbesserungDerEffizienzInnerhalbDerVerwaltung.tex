\subsection{Verbesserung der Effizienz innerhalb der Verwaltung}
Die Verwaltungsmodernisierung hat das Ziel, die Effizienz, Wirksamkeit und Bürgernähe der Verwaltung zu verbessern.
Ein Ansatz zur Verwaltungsmodernisierung ist das Smart Government, das die Verwaltung durch den Einsatz von Informationstechnologie und digitalen Lösungen verbessern soll.
Beim Smart Government bewegen sich viele Anwendungsfelder in der öffentlichen Verwaltung auf einem schmalen Grat zwischen den Chancen, die Smart Government mitbringt und Risiken, wie Datenschutz, Überwachung und Fremdsteuerung \citep[Vgl.][]{Lucke2018}.
\par
Ein Ziel, das beim Ansatz des Smart Governments zu betrachten ist, ist die Verbesserung der Effizienz.
Durch den Einsatz von IT-Lösungen, Datenanalyse und automatisierten Verfahren kann Smart Government die Verwaltungsprozesse beschleunigen und standardisieren. 
Ziel ist es die papierbasierten Abläufe durch elektronische Akten- und Vorgangsbearbeitungssysteme abzulösen, um die Effizienz zu steigern \citep[][]{von_Lucke_2016}.
\par
So liegt laut von Lucke das größte Potential zur Verbesserung der Effizienz darin, nicht die Eins-zu-eins-Umsetzung von papierbasierten Abläufen in intelligentes Papier, sondern die komplette Überarbeitung dieser Vorgänge in rein digitale Verwaltungsprozesse \citep[][S.179]{von_Lucke_2016}.
Das Auslagern dieser Prozesse in das solche Akten- und Vorgangsbearbeitungssysteme ermöglicht eine stärkere und schnelle Massenbearbeitung von Anträgen, Rechnungen und Genehmigungsprozessen, was eine deutliche Erhöhung der Effizienz bedeutet.
\par
Zusammenfassend trägt Smart Government dazu bei, die Effizienz innerhalb der Verwaltung zu verbessern. 
Durch den Einsatz von IT-Lösungen und automatisierten Verfahren kann die Verwaltung effizienter, transparenter und bürgernäher agieren.