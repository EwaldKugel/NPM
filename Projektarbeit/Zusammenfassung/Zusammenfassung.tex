\section{Zusammenfassung}
In dieser Arbeit wurde zunächst das Konzept Smart Government vorgestellt und die möglichen Chancen und Herausforderungen für die öffentliche Verwaltung beleuchtet.
Anschließend galt es, diese Form der Verwaltungsmodernisierung von anderen Konzepten und Modellen abzugrenzen und die unterschiedlichen Einflüsse auf Effizienz und Bürgerbeteiligung zu vergleichen.

Die Besonderheit von Smart Government ist die Anwendung aktueller Technologien aus dem Spektrum des Internets der Dinge mit dem Ziel Verwaltungsdienstleistungen effizienter und bürgerfreundlicher bereitstellen zu können.
So wird die Verwendung von künstlicher Intelligenz zur Lösung von Routineaufgaben oder die Verwendung von smarten Gegenständen beispielsweise zur Warnung der Bevölkerung (smarte Bojen) vorgeschlagen.
Dabei steht mit der Anwendung von Prinzipen wie Open Data und Open Government die konsequente Vernetzung von Bürgern und der Verwaltung im Vordergrund.

Damit setzt Smart Government einen Weg fort, der in Deutschland durch die beiden anderen Methoden zur Veraltungsmodernisierung die in dieser Arbeit vorgestellt wurden bereits seit den Neunziger Jahren sukzessive beschritten wird.
Durch die breite Anwendung von Elementen des Neuen Steuerungsmodells hat sich die grundsätzliche Ausrichtung der kommunalen, aber im Laufe der Zeit auch der Landes- und Bundesbehörden, auf den Bürger gerichtet.
So wurden zunächst durch die Annäherung innerer Strukturen an die Privatwirtschaft die Voraussetzungen geschaffen, Verwaltung als Dienstleistung am Bürger zu verstehen.
Mit der Einführung von E-Government wurde der nächste Schritt gemacht, indem Prozesse nicht nur digitalisiert werden, sondern gleichzeitig dem Bürger die Möglichkeit gegeben wird, diese auch digital in Anspruch zu nehmen.

Mit dem Einsatz von moderner Technologie, speziell über das Internet, gilt es jedoch auch einige Herausforderungen und Risiken zu Betrachten.
Dazu zählen zum einen die hohen Kosten die mit der Umstellung auf eine digitale Verwaltung und dem Aufbau der dazu benötigten Infrastruktur einhergehen.
Gerade im Bereich der Daten- und Informationsverarbeitung und der Vernetzung, die im Zuge der Anwendung von Smart Government so viele Vorteile mit sich bringt, ist der Schutz sensibler Daten sicherlich die größte Herausforderung.

Abschließend lässt sich festhalten, dass Verwaltungsmodernisierung ein Prozess ist, der fortlaufend neu erdacht wird.
Dabei spielen Einflüsse aus der Privatwirtschaft und gerade in diesem Jahrtausend technische Entwicklungen eine besondere Rolle.
Der Einsatz modernster Technik in der Verwaltung ist jedoch nicht trivial, da mitunter Gesetze geändert müssen um beispielsweise die rechtssichere Kommunikation mithilfe elektronischer Mittel sicherzustellen.
Die Vorteile, sowohl für den Bürger als auch für die Mitarbeiter in der Verwaltung, die durch die verschiedenen Elemente von Smart Government in Aussicht gestellt werden sind dabei nicht von der Hand zu weisen.
Damit liegt es an der Gesetzgebung, wie im Fall von E-Government die rechtlichen Grundlagen zu schaffen, um sicher von diesen Vorteilen provitieren zu können und eine konsequente Weiterentwicklung der Verwaltung zu ermöglichen.

% Gleichzeitig 
% 
% 
% Trotz der vielen Vorteile birgt Smart Government auch einige Risiken. 
% Eines der größten Risiken ist die Cyber-Sicherheit. Da die Verwaltungsprozesse zunehmend digitalisiert werden, müssen sicherheitsrelevante Aspekte wie Datenschutz und Informationssicherheit berücksichtigt werden, um zu verhindern, dass Daten in falsche Hände gelangen. 
% Darüber hinaus kann die Überführung von Verwaltungsprozessen in die digitale Welt für einige Bürger als Überforderung empfunden werden.
% 
% Im Vergleich zu anderen Verwaltungsmodernisierungen zeichnet sich Smart Government durch seinen starken Fokus auf die Verwendung von Technologie aus. 
% Im Gegensatz zu traditionellen Verwaltungsmodernisierungen, die hauptsächlich auf organisatorische Veränderungen abzielen, setzt Smart Government auf die Nutzung digitaler Lösungen, um Prozesse zu optimieren und bessere Dienstleistungen bereitzustellen.
% 
% Zusammenfassend lässt sich sagen, dass Smart Government eine Vielzahl von Chancen und Herausforderungen birgt. 
% Obwohl es eine effektivere Verwaltung und bessere Dienstleistungen ermöglichen kann, müssen die Risiken im Hinblick auf Cyber-Sicherheit und möglicher Überforderung