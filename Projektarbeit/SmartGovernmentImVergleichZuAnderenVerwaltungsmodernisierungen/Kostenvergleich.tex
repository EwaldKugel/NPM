\subsection{Kostenvergleich}
Eine der wichtigsten Überlegungen bei der Implementierung neuer Technologien in Verwaltungen ist die Kosteneffizienz.
In diesem Unterkapitel werden wir uns auf den Vergleich der Kosten zwischen Smart Government und anderen Verwaltungsmodernisierungsarten in Deutschland konzentrieren.
\par
Zu den Kosten, die bei der Implementierung von Smart Government berücksichtigt werden müssen, gehören sowohl die Anschaffungskosten für die notwendige Technologie als auch die laufenden Kosten für Wartung und Betrieb.
Außerdem müssen eventuelle Umstellungskosten auf die neue Technologie ebenfalls berücksichtigt werden.
Andere Verwaltungsmodernisierungsarten, wie beispielsweise die Automatisierung von Prozessen oder die Digitalisierung von Dokumenten, haben ebenfalls Kosten, die es zu berücksichtigen gilt.
\par
Um einen Vergleich zu ermöglichen, müssen jedoch die Vorteile, die jede Verwaltungsmodernisierung bietet, ebenfalls berücksichtigt werden.
So kann Smart Government beispielsweise durch die Integration von Daten und Prozessen zu einer effizienteren Verwaltung beitragen, was wiederum zu Einsparungen in der Verwaltung führen kann.
Andere Verwaltungsmodernisierungsarten können ebenfalls zu Einsparungen führen, beispielsweise durch die Reduzierung von Papier- und Druckkosten.
\par
Die Analyse der Kosten ist jedoch nicht einfach, da es schwierig ist, die Vorteile jeder Verwaltungsmodernisierung genau zu quantifizieren.
Trotzdem ist es wichtig, einen Vergleich anzustellen, um die besten Optionen für die Verwaltung zu identifizieren.
In diesem Unterkapitel werden wir daher eine kritische Bewertung der Kosten von Smart Government im Vergleich zu anderen Verwaltungsmodernisierungsarten vornehmen.