\subsection{Überblick über andere Modelle zur Verwaltungsmodernisierung}
Grundsätzlich liegt das Ziel bei der Modernisierung von Verwaltung heutzutage darin, vorhandene Strukturen aufzubrechen, Prozesse zu optimieren und für den Bürger zugänglicher zu machen.
Das hat je nach Perspektive und verfügbarer Technologie in den letzten Jahrzehnten zu unterschiedlichen Ansätzen und Entwicklungen geführt.
In Deutschland wurden und werden zwei wichtige Ansätze zur Verwaltungsmodernisierung verfolgt.
Im Folgenden sollen die Konzepte \glqq{}Neues Steuerungsmodell\grqq{} und \glqq{}E-Government\grqq{} kurz vorgestellt werden.

\textbf{Neues Steuerungsmodell}\\%Werner Jann Neues Steuerungsmodell, Zusammenfassung
Das Neue Steuerungsmodell ist ein Ansatz zur Verwaltungsmodernisierung, der auf den Prinzipien von Steuerung und Regulierung basiert. 
Es zielt darauf ab, die Effizienz und Wirksamkeit der Verwaltung zu erhöhen, indem klare Ziele, Verantwortlichkeiten und Überwachungsmechanismen eingeführt werden.
Das Neue Steuerungsmodell wird auch von Jann als die \glqq{}spezifisch deutsche Version\grqq{} \citep[vgl.][S. 127]{Jann2019} des \glqq{}New Public Managements\grqq{} bezeichnet.
Bei beiden Modellen geht es darum, den bürokratischen Verwaltungscharakter abzustreifen und öffentliche Verwaltungen zu Dienstleistungsunternehmen umzubauen.
In Deutschland bezieht sich das in erster Linie auf die kommunale Ebene und hat seine Ursprünge im den frühen 1990er Jahren.
Dabei finden einige Leitlinien aus der Privatwirtschaft Anwendung, wie die Leistungsorientierung, Produktorientierung oder Kundenorientierung.
Auf die Zielsetzung bezogen können laut Jann so drei Kernelemente des NSM \citep[vgl.][S. 127]{Jann2019} ausgemacht werden:

\begin{itemize}
  \item Aufbau einer dezentralen Führungs- und Organisationsstruktur
  \item Steuerung der Verwaltung durch politische Kontrakte
  \item Wettbewerb und Kundenorientierung
\end{itemize}

Voraussetzung für eine effektive Steuerung der Haushaltsmittel unter Berücksichtigung der Produktorientierung ist jedoch eine Haushaltsführung, die die zeitliche Entwicklung abbildet.
Mit der traditionellen kameralistischen Haushaltsführung der Kommunen kann in erster Linie der Ist-Zustand eines Jahres, bezogen auf Einnahmen und Ausgaben, festgestellt werden.
Durch die Umstellung auf die in der privatwirtschaft übliche bzw. verpflichtende doppelte Buchführung, die sog. Doppik, wird die Beurteilung der aktuellen Situation unter Berücksichtigung von Vermögenswerten ermöglicht.
So können insgesamt verlässlichere Entscheidungen aufgrund fundierter Informationen getroffen werden, wobei beispielsweise Kredite und zukünftige Pensionen mit einbezogen werden.
Diese Umstellung wurde jedoch erst 2003 im Rahmen des Neuen kommunalen Finanzmanagements von der Innenministerkonferenz angestoßen, wobei es keine bundeseinheitliche Regelung gibt \citep[][]{Bogumil2017}.
So nutzen im Jahr 2017 nur 7.748 von 11.927 deutschen Kommunen die Doppik, während Bayern und Thüringen auf eine erweiterte Kameralistik setzen \citep[][]{Burth2017}.
Diese zeichnet sich hauptsächlich durch eine zusätzliche Kosten- und Leistungsrechnung aus, um den Zielen des NSM gerecht zu werden.
Im Kern handelt es sich jedoch weiterhin um die klassische Kameralistik, die auf unterschiedliche Elemente der Betriebswirtschaft zurückgreift.

Die produktorientierte Steuerung in Verbindung mit einer dezentralen Organisationsstruktur soll in erster Linie die Prozesse innerhalb der Verwaltung beschleunigen.
Durch die Zuweisung eines Budgets mit Bindung an vereinbarte Ziele entfällt die Verhandlung und Genehmigung von Einzelfällen.
Das heißt, die Fachbereiche können in Eigenverantwortung effektiv und effizient handeln.
Um diesen Prozess voranzutreiben wird das Mittel des Wettbewerbs zwischen den Kommunen anhand von Kennzahlen eingeführt.
So soll eine stetige Verbesserung der Prozesse angetrieben werden.

\textbf{E-Government}\\
E-Government meint laut Bundesregierung den \glqq{}Einsatz elektronischer Informationstechnologien, damit Verwaltungsangebote für jedermann einfach, schnell und ortsunabhängig zugänglich sind\grqq{} \citep[][]{Bundesregierung2023}.
Dabei liegt der Fokus darauf, elektronische Entsprechungen bereits vorhandener Verwaltungsangebote sowie die dazu notwendigen Voraussetzungen zu schaffen.

Die Voraussetzzungen für den Einsatz von elektronischer Informationstechnologie im Rahmen von Verwaltungsdienstleistungen wurden mit dem E-Government-Gesetz (2013) geschaffen.
Zentrale Punkte dieses Gesetzes sind die Abschaffung von Schriftformerfordernissen und Erfordernissen zur persönlichen Vorsprache in verschiedenen Fachgesetzen.
In Verbindung mit der Einrichtung von De-Mail als elektronische Schriftform, sowie der Möglichkeit der elektronischen Identifizierung durch die eID-Funktion des Personalausweises, können so viele Dienstleistungen elektronisch angeboten werden \citep[][]{BMI2023}.
Durch die Verabschiedung des Onlinezugangsgesetzes (2017) wurden Bund, Länder und Kommunen zur Bereitstellung eben dieser Dienstleistungen verpflichtet.
Grundlegendes Ziel ist dabei die Errichtung eines Portalverbunds, in dem sich Bürger mit einem Nutzerkonto anmelden und anschließend alle verfügbaren Verwaltungsdienstleistungen von Bund, Ländern und Kommunen nutzen können.

Die Kombination dieser Gesetze bildet die Anwendung von E-Government als Methode zur Verwaltungsmodernisierung in Deutschland ab und zeigt deutlich das Hauptziel von E-Government auf.
Durch die Ersetzung von Schriftformerfordernissen und persönlicher Vorsprache durch elektronische Entsprechungen entfällt der Gang zur Behörde für viele Verwaltungsdienstleistungen.
So können Prozesse effizienter gestaltet und Bürokratie abgebaut werden.