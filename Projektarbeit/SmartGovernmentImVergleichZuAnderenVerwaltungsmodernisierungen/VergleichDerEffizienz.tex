\subsection{Betrachtung der unterschiedlichen Einflüsse auf die Effizienz der Verwaltung}
Mit Effizienz ist hier die wirtschaftliche Betrachtung der verwendeten Ressourcen in Bezug auf das damit erzielte Ergebnis gemeint.
Auf die Verwaltung bezogen gibt es verschiedene Gesichtspunkte die zu einer effizienten Verwaltung beitragen können.
Im weiteren Verlauf soll der (mögliche) Einfluss der vorgestellten Methoden zur Verwaltungsmodernisierung auf die Einführung schlanker und effizienter Prozesse untersucht werden.

Der wohl wichtigste Faktor bei der Untersuchung der Effizienz der öffentlichen Verwaltung ist die Betrachtung der einzelnen Arbeitsabläufe.
Eine konkrete Messbarkeit für die Effizienz von Prozessen bedingt eine genaue Aufstellung der für das Erreichen eines bestimmten Ergebnis aufgewendeten Mittel.
Mit der Einführung des NSM und der Umstellung auf einen produktorientierten Haushalt kann die Voraussetzung dafür geschaffen werden.
Denn erst durch Wettbewerb und den Vergleich von Kennzahlen, die eng mit der Effizienz verbunden sind, besteht die Notwendigkeit zu effizientem Verwaltungshandeln \cite[][]{Jann2019}.
Vor allem der Aufbau einer dezentralen Organisationsstruktur und Ressourcenverantwortung führt zu kürzeren Wegen innerhalb der Organisation, da viele Entscheidungen nicht mehr von oberster Ebene abgesegnet werden müssen.

Die Digitalisierung der Prozesse im Zuge von E-Government kann darauf aufbauen und zusätzliche Zeitersparnisse erreichen.
Beispielsweise werden für digitale Prozesse keine Umlaufmappen benötigt, sodass der nächste Prozessschritt unmittelbar folgen kann.
Eine wichtige Voraussetzung hierbei ist jedoch, dass die Prozesse vollständig frei von Medienbrüchen sind.
Dazu sind einheitliche Schnittstellen und Formate nötig, zunächst innerhalb einer Behörde, für weitere Optimierung aber mitunter landes- oder bundesweit.

Durch die Öffnung der Prozesse und gleichzeitige Einbindung des Bürgers in den Prozessablauf können weitere Verbesserungen erreicht werden.
Dabei helfen standardisierte Eingabemasken sowie die Übermittlung erforderlicher Unterlagen in digitaler Form.
Ohne die Notwendigkeit der persönlichen Vorsprache wird dem Sachbearbeiter die Arbeit abgenommen, die (E-)Akte zu erstellen und eventuell Dokumente zu kopieren oder zu digitalisieren.
Dies passiert bei der elektronischen Antragstellung durch geführte Formulare, in denen der Bürger die erforderlichen Daten hinterlegt und Dokumente hochladen kann.
Im Idealfall erhält der Sachbearbeiter eine vollständige Akte und kann den Sachverhalt direkt beurteilen, zumindest aber wird eine Akte mit Kerndaten angelegt und es müssen eventuell benötigte Unterlagen nachgefordert werden.

Durch die Automatisierung von (Teil-)Prozessen mittels künstlicher Intelligenz stellt Smart Government den Wegfall von Routineaufgaben in Aussicht.
Damit würden Mitarbeiter entlastet und könnten sich beispielsweise auf den Erlass eines Verwaltungsaktes konzentrieren statt auf das Einholen aller benötigten Informationen.
Im oben konstruierten Beispiel würde der Bürger in einem entsprechenden elektronischen Formular einwilligen, dass bestimmte bei anderen Behörden oder Institutionen hinterlegte Daten automatisch eingeholt und in den Prozess miteinbezogen werden.
Dies wäre eine weitere Entlastung für alle Prozessteilnehmer, inklusive dem Bürger.
So könnten beispielsweise zusätzliche Anträge entfallen die für die Einholung von Informationen notwendig sein können, da diese im Hintergrund durch die Verwendung gemeinsamer Datenbanken bereits zur Verfügung stehen.
Beispiele dafür gibt es bereits in anderen Ländern, siehe Kapitel~\ref{SGEffizienz}.

Zusätzlich ist die Anwendung von Errungenschaften aus dem Internet der Dinge und Dienste ein wichtiger Schritt zu verbesserten Prozessen, aber auch zu neuen Möglichkeiten \citep[][]{Lucke2016}.  
Ein Beispiel für die Anwendung von Sensorik in Verbindung mit smarter Software (neuronale Netze) nennt Schuppan.
Mithilfe dieser Technologie wäre es möglich Autobahnbrücken zu überwachen, Sanierungsbedarfe rechtzeitig zu erfassen und finanzielle Mittel dafür planen zu können \citep[][]{Schuppan2016}.
Ein anderes Beispiel ist die Warnung vor Tsunamis durch smarte Bojen \citep[][]{Lucke2016}.
Hier werden vor allem auch Prozesse möglich, die ohne die Einbindung von smarten Geräten nur schwer bis garnicht durchführbar wären.

Zusammengefasst lässt sich festhalten, dass die Optimierung der Prozesse beim E-Government und Smart Government vor allem auf technischer Ebene stattfindet, während beim NSM strukturelle Änderungen der Organisation zu verbesserten Prozessen führt.
