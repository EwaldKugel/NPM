\subsection{Betrachtung der unterschiedlichen Einflüsse auf die Effizienz der Verwaltung}
Mit Effizienz ist hier die wirtschaftliche Betrachtung der verwendeten Ressourcen in Bezug auf das damit erzielte Ergebnis gemeint.
Auf die Verwaltung bezogen gibt es verschiedene Gesichtspunkte die zu einer effizienten Verwaltung beitragen können.
Im weiteren Verlauf soll der (mögliche) Einfluss der vorgestellten Methoden zur Verwaltungsmodernisierung auf einige dieser Punkte untersucht werden.

\textbf{Schlanke und effiziente Prozesse}\\
Der wohl wichtigste Faktor bei der Untersuchung der Effizienz der öffentlichen Verwaltung ist die Betrachtung der einzelnen Arbeitsabläufe.
Eine konkrete Messbarkeit für die Effizienz von Prozessen bedingt eine genaue Aufstellung der für das Erreichen eines bestimmten Ergebnis aufgewendeten Mittel.
Mit der Einführung des NSM und der Umstellung auf einen produktorientierten Haushalt kann die Voraussetzung dafür geschaffen werden.
Denn erst durch Wettbewerb und den Vergleich von Kennzahlen, die eng mit der Effizienz verbunden sind, besteht die Notwendigkeit zu effizientem Verwaltungshandeln \cite[][]{Jann2019}.
Vor allem der Aufbau einer dezentralen Organisationsstruktur und Ressourcenverantwortung führt zu kürzeren Wegen innerhalb der Organisation, da Entscheidungen nicht mehr von oberster Ebene abgesegnet werden müssen.

Die Digitalisierung der Prozesse im Zuge von E-Government kann darauf aufbauen und zusätzliche Zeitersparnisse erreichen.
Beispielsweise werden für digitale Prozesse keine Umlaufmappen benötigt, sodass der nächste Prozessschritt unmittelbar folgen kann.
Eine wichtige Voraussetzung hierbei ist jedoch, dass die Prozesse vollständig frei von Medienbrüchen sind.
Dazu sind einheitliche Schnittstellen und Formate nötig, zunächst innerhalb einer Behörde, für weitere Optimierung aber mitunter landes- oder bundesweit.
Die Optimierung der Prozesse findet hier vor allem auf technischer Ebene statt, während beim NSM strukturelle Änderungen der Organisation zu verbesserten Prozessen führt.

Durch die Automatisierung von (Teil-)Prozessen mittels künstlicher Intelligenz stellt Smart Government den Wegfall von Routineaufgaben in Aussicht.
Damit würden Mitarbeiter entlastet und könnten sich beispielsweise auf den Erlass eines Verwaltungsaktes konzentrieren statt auf das Einholen aller benötigten Information.
Zusätzlich ist die Anwendung von Errungenschaften aus dem Internet der Dinge und Dienste ein wichtiger Schritt zu verbesserten Prozessen, aber auch zu neuen Möglichkeiten \citep[][]{Lucke2016}.  
Ein Beispiel für die Anwendung von Sensorik in Verbindung mit smarter Software (neuronale Netze) nennt Schuppan.
Mithilfe dieser Technologie wäre es möglich Autobahnbrücken zu überwachen, Sanierungsbedarfe rechtzeitig zu erfassen und finanzielle Mittel dafür planen zu können \citep[][]{Schuppan2016}.
Ein anderes Beispiel ist die Warnung vor Tsunamis durch smarte Bojen \citep[][]{Lucke2016}.
Hier werden vor allem auch Prozesse möglich, die ohne die Einbindung von smarten Geräten nur schwer bis garnicht durchführbar wären.

\textbf{Kostenbetrachtung}\\


% Die Effizienz einer Verwaltungsmodernisierung wird oftmals anhand verschiedener Faktoren wie Prozessbeschleunigung, Servicequalität und Kosteneinsparungen bewertet.
% Im Vergleich zu anderen Verwaltungsmodernisierungsarten zeichnet sich Smart Government besonders durch seine innovative und integrative Technologie aus.
% \par
% Smart Government nutzt moderne Informations- und Kommunikationstechnologien, um den Informationsfluss zwischen Bürgern, Unternehmen und Verwaltungen zu optimieren und den Zugang zu Dienstleistungen zu erleichtern.
% Durch die Verwendung von digitalen Tools und die Integration von Daten kann Smart Government auch den internen Prozessablauf in Verwaltungen verbessern.
% Hierdurch wird eine höhere Effizienz im Vergleich zu herkömmlichen Verwaltungsmodernisierungsarten erreicht.
% \par
% Eine Studie des Fraunhofer-Instituts für System- und Innovationsforschung (ISI) belegt, dass Smart Government eine höhere Effizienz im Vergleich zu traditionellen Verwaltungsmodernisierungsarten aufweist.
% So konnten durch die Verwendung von Smart Government Lösungen beispielsweise Prozesse beschleunigt, Kosten reduziert und die Servicequalität verbessert werden.
% \par
% Allerdings ist es wichtig zu beachten, dass eine effiziente Umsetzung von Smart Government auch von der Qualität der Implementierung und der technischen Infrastruktur abhängt.
% Eine schlechte Umsetzung kann zu Problemen bei der Datenintegration und einer nicht ausreichenden Nutzerfreundlichkeit führen, was wiederum zu einer geringeren Effizienz führt.
% \par
% Zusammenfassend lässt sich sagen, dass Smart Government eine hohe Effizienz im Vergleich zu anderen Verwaltungsmodernisierungsarten aufweist, wenn es effektiv implementiert wird.
% Um die Effizienz von Smart Government optimal nutzen zu können, müssen Unternehmen und Verwaltungen jedoch auch in eine angemessene technische Infrastruktur investieren.
%Smart Government: nicht Papierverfahren 1:1 in digitale übersetzen, sondern Überarbeitung beim Digitalisieren (gleichzieit verbessern)