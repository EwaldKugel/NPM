\subsection{Anteilnahme des Bürgers bei verschiedenen Modellen}
Die Beteiligung der Bürger bei der Verwaltung ist ein wichtiger Indikator für den Erfolg einer Verwaltungsmodernisierung.
Im Zuge der Digitalisierung haben sich viele Regierungen zu einer stärkeren Einbindung der Bürger bei Entscheidungsprozessen und der Bereitstellung von Dienstleistungen durch Smart Government bekannt.
Doch wie steht es um die Bürgerbeteiligung im Vergleich zu anderen Verwaltungsmodernisierungen in Deutschland?
\par
Eine Möglichkeit, die Bürgerbeteiligung zu messen, ist die Nutzung von Online-Tools und Plattformen, die es den Bürgern ermöglichen, an Entscheidungsprozessen teilzunehmen und Feedback zu geben.
Studien zeigen, dass Smart Government-Projekte in Deutschland eine höhere Nutzung solcher Tools aufweisen, was auf eine höhere Bürgerbeteiligung hinweist.
\par
Ein weiteres Indiz für eine hohe Bürgerbeteiligung ist die Zufriedenheit mit den bereitgestellten Dienstleistungen.
Smart Government-Projekte haben in der Regel eine hohe Zufriedenheit bei den Bürgern aufgewiesen, was auf eine erfolgreiche Integration der Bürger in den Entscheidungsprozess und die Bereitstellung von Dienstleistungen hindeutet.
\par
Andere Verwaltungsmodernisierungen, wie beispielsweise die Einführung von Papierlosen Büros, haben jedoch auch zu einer höheren Bürgerbeteiligung beigetragen.
Durch den Einsatz von Online-Systemen und digitalen Signaturen können Bürger ihre Angelegenheiten einfacher und schneller regeln, ohne dass sie persönlich in der Verwaltung erscheinen müssen.
\par
Insgesamt lässt sich sagen, dass Smart Government eine hohe Bürgerbeteiligung aufweist, aber auch andere Verwaltungsmodernisierungen zu einer höheren Bürgerbeteiligung beitragen können.
Es ist wichtig, dass Regierungen auf die Bedürfnisse und Wünsche der Bürger eingehen, um eine erfolgreiche Verwaltungsmodernisierung umzusetzen.