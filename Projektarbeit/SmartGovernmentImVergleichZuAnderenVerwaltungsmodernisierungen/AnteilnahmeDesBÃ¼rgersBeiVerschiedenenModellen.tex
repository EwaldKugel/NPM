\subsection{Anteilnahme des Bürgers bei verschiedenen Modellen}
Wie bereits in Kapitel~\ref{VortBürgerbeteiligung} dargelegt, stellt die Verbesserung der Bürgerbeteiligung ein hohes Qualitätsmerkmal für die Verwaltungsmodernisierung dar.
Mit der Ausrichtung von Verwaltung als Dienstleister für den Bürger hat die Ermöglichung und Vereinfachung von Bürgerbeteiligung immer mehr an Bedeutung gewonnen.
In diesem Kapitel werden die vorgestellten Modelle zur Verwaltungsmodernisierung auf ihre Methoden zur Bürgerbeteiligung untersucht.

Das NSM schafft durch die grundsätzliche Ausrichtung auf den Bürger auch hier die Grundlage.
Zwar werden keine konkreten Maßnahmen zur Einbeziehung der Bürger vorgeschlagen, allerdings muss gerade in den 90er und 2000er Jahren zunächst ein bürgerbezogenes Bewusstsein innerhalb der Verwaltung entwickelt werden.

Die Digitalisierung ermöglicht im Rahmen von E-Government auch die Einführung elektronischer Angebote zur Bürgerbeteiligung.
Dabei erfahren vor allem traditionelle Abstimmungsvarianten und Kooperationsverfahren eine Aufwertung durch elektronische Modelle \citep[][]{Leitner2018}.
Die daraus resultierenden Angebote werden unter dem Schlagwort E-Partizipation zusammengefasst und beinhalten mehrere Intensitätsstufen, die sich in ihrer Verbindlichkeit auszeichnen.

So geht es bei der ersten Stufe um die Bereitstellung von Informationen beispielsweise durch Newsletter oder Internetauftritte. 
Daneben sind aber auch Elemente von Open Data und Open Government gemeint, wie die Bereitstellung von Daten in öffentlichen Datenbanke oder die Übertragung von parlamentarischen Sitzungen.

In der zweiten Stufe (Konsultation) wird der Bürger mit in die Kommunikation einbezogen, so sind digitale Diskussionsforen oder ein Beschwerdemanagement vorstellbar bei dem beispielsweise über eine mobile App Schlaglöcher, Graffitis o.Ä. gemeldet werden können.
Ein konkretes Beispiel ist das Formular \glqq{}Mängelmeldung\grqq{} der Stadt Gladbeck\footnote{verfügbar unter\\ \url{https://www.gladbeck.de/Rathaus_Politik/Rathaus/Online-Services/_Maengelmeldung/_Neuer_Vorgang.asp}}, über das eine genaue Beschreibung inklusive Bildern möglich ist.

Die dritte Stufe unterscheidet zwischen Angeboten zur Kooperation sowie Mitentscheidung.
Bei der Kooperation geht es zum Beispiel darum, Bürger bei der Erstellung von Beschlussvorlagen oder bei der Stadtentwicklungsplanung aktiv mit einzubeziehen.
Das kann durch die Möglichkeit der Mitarbeit an gemeinsamen Dokumenten geschehen oder durch Einreichung von Ideen und Bedenken über das Internet. 
Bei der Mitentscheidung als intensivste Stufe können, unter Beachtung rechtlicher Voraussetzungen, durch E"~Voting, Online-Umfragen oder E"~Petitionen Grundlagen für Beschlüsse mitbestimmt werden \citep[vgl.][]{Leitner2018}.

Die Hauptfaktoren bei allen Angeboten zur E-Partizipation sind eine möglichst niedrige Beteiligungshürde bei gleichzeitiger sicherer Identifikation.
Entsprechende Instrumente müssen im Zuge der Verwaltungsmodernisierung durch E-Government bereitgestellt werden (vgl. Kapitel~\ref{Ueberblick}).

Smart Government zielt in diesem Zusammenhang auf die konsequente Weiterentwicklung von E-Partizipation unter verstärkter Einbindung von Open Data und Open Government Strategien ab.






%Smart Government: Portal für Services, Zugriff auf Infos der Verwaltung (Transparenz führt zu Vertrauen), Kommunikation, Datenbanken (Open Data, Government)
%Bürgerbeteiligung als Qualitätsmerkmal für Verewaltungsmodernisierung (ausgeprägt durch online Volksabstimmungen)