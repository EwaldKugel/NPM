\subsection{Abgrenzung der vorgestellten Modelle}
Zur Abgrenzung der Verwaltungsmodernisierungsmodelle Neues Steuerungsmodell, E-Government und Smart Government werden diese anhand einiger Kernelemente der vorgestellten Modelle gegenübergestellt.
Dabei gibt es offensichtlich grundlegende Unterschiede zwischen NSM und den beiden \glqq{}modernen\grqq{} Methoden E-Government und Smart Government, während letztere thematisch deutlich näher zusammen liegen.

\textbf{Steuerung der Maßnahmen}\\
Die Entwicklung des NSM ging hauptsächlich von den Kommunen aus und es gab lediglich eine Empfehlung zur Einführung der Doppik als zentralen Baustein von der Innenministerkonferenz 2003.
Das führte in der Folge zu unterschiedlichen Ausgestaltungen der Länder was die Pflicht zur Einführung aber auch die Fristen betrifft \citep[][]{Mehde2019}.
Im Gegensatz dazu wird die Einführung des E-Governments durch Bundesgesetze gesteuert wodurch Länder und Kommunen zum einen befähigt, im Anschluss aber auch dazu verpflichtet werden bspw. Dienstleistungen elektronisch zur Verfügung zu stellen.
Diese Gesetze (speziell das E-Government-Gesetz) ermöglichen zusätzlich die Anwendung von Elementen aus dem Bereich Smart Government.
Speziell sei hier der Bereich Open Data genannt.

\textbf{Adressat der Maßnahmen}\\
%NSM intern
Zwar hat sich mit dem NSM der Fokus der Verwaltung auf den Bürger als Kunden gerichtet, die Konsequenzen in der Gestaltung zielen jedoch hauptsächlich auf interne Veränderungen struktureller Natur ab.
So hat die Umstellung des Haushalts und die Einführung einer dezentralen Organisationsstruktur unmittelbar keine wahrnehmbaren Auswirkungen für den Bürger.

%E-Government intern erleichterung durch Effizienz (Risiken beachten), Bürger + Wirtschaft + andere Behörden
Mit der Einführung von E-Government ändert sich das.
Zwar ist auch hier der erste Schritt eine interne Modernisierung der Prozesse, konkret die Digitalisierung, im Anschluss werden diese Prozesse jedoch zusätzlich für den Bürger geöffnet.
Dazu werden Schnittstellen zur Verfügung gestellt über die der Bürger Anträge oder Materialien wie Nachweise elektronisch an die Behörde senden kann.
Dabei steht die Verhinderung eines Medienbruchs zur Verbesserung der Effizienz und zum Abbau von Bürokratie im Vordergrund.
Neben dem Bürger profitieren zusätzlich Firmen aus der Privatwirtschaft von der neuen Art der Kommunikation mit Behörden, beispielsweise bei der Zusammenarbeit im Rahmen öffentlicher Verträge.

%Smart Government Bürger und mehr Behörden und Wirtschaft
Dieser Trend setzt sich im Smart Government fort.
Hier spielt zusätzlich die Zusammenarbeit von Behörden eine große Rolle, beispielsweise bei der Informationsbeschaffung für die Bewertung von Anträgen.
Mithilfe gemeinsamer Datenbanken und automatisierter (Teil-)Prozesse können so Vorgänge beschleunigt und Bürokratie abgebaut werden.

\textbf{Auswirkung auf Prozesse}\\
%NSM interne Umstellung, strukturelle Veränderung
Die Einführung einer dezentralen Organisationsgestaltung sowie die Umstellung auf einen produktorientierten Haushalt haben große strukturelle Auswirkungen auf die Arbeitsweise innerhalb einer Kommune.
So werden plötzlich Entscheidungen innerhalb eines Fachbereichs getroffen, was neue Herausforderungen an Führungskräfte aber auch die Mitarbeiter nach sich zieht.
Insbesondere der Wettbewerb (über Kennzahlen) mit anderen Kommunen führt im weiteren Verlauf zu Anpassungen der veränderten Prozesse.
Dabei steht stets das Ziel im Vordergrund, diese Prozesse ständig weiterzuentwickeln und effizienter zu gestalten.

%E-Government digitalisierung der Prozesse inkl. Anbieten einer Schnittstelle für den Bürger
Die Einführung von E-Government hat nicht weniger große Auswirkungen auf die Prozesse innerhalb der Verwaltung.
Jedoch beziehen sich diese Auswirkungen nicht auf grundsätzliche Änderungen von Abläufen sondern auf den technischen Aspekt.
So werden bestehende, zum Teil in Papierform vollzogene, Arbeitsabläufe digitalisiert.
Herausforderungen entstehen dabei weniger auf struktureller Ebene als vielmehr im Umgang mit entsprechender Software wie betrieblichen Informationssystemen.

%Smart Government automatisierung durch KI, Big data
Durch den Einsatz von künstlicher Intelligenz und behördenübergreifend verfügbaren Daten geht der Smart Government Ansatz noch einen Schritt weiter.
Ausdrücklich wird hier neben der Digitalisierung auch eine Optimierung der Prozesse angestrebt, wobei zunächst Routineaufgaben vereinfacht bzw. automatisiert werden sollen.
Dadurch könnten mithilfe umfassender Daten und intelligenten Algorithmen objektive Grundlagen geschaffen werden, die bei Ermessensentscheidungen helfen können \citep[][]{Demaj2018}.
Gleichzeitig wird eine weitere Öffnung der Prozesse im Sinne des Open Governments angestrebt, sodass sich Bürger und Unternehmen selbstständig über den Fortschritt von Anträgen informieren können.



