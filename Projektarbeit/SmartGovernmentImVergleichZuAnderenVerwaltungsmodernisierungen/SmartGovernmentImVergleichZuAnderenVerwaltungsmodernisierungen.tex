\section{Smart Government im Vergleich zu anderen Verwaltungsmodernisierungen}
In diesem Kapitel werden zunächst die aktuell in Deutschland relevanten Modelle zur Verwaltungsmodernisierung vorgestellt.
Dabei wird insbesondere auf die nötigen Voraussetzungen und Maßnahmen eingegangen.
Anschließend werden die in dieser Arbeit behandelten Modelle voneinander abgegrenzt und (mögliche) Einflüsse dieser Modelle auf zwei Kernpunkte der öffentlichen Verwaltung diskutiert.
Dabei wird die Effizienz innerhalb der Verwaltung betrachtet, bevor auf die Kommunikation mit und die Beteiligung von Bürgern an der Verwaltung behandelt wird.
\subsection{Überblick über andere Modelle zur Verwaltungsmodernisierung}\label{Ueberblick}
Grundsätzlich liegt das Ziel bei der Modernisierung von Verwaltung heutzutage darin, vorhandene Strukturen aufzubrechen, Prozesse zu optimieren und für den Bürger zugänglicher zu machen.
Das hat je nach Perspektive und verfügbarer Technologie in den letzten Jahrzehnten zu unterschiedlichen Ansätzen und Entwicklungen geführt.
In Deutschland wurden und werden zwei wichtige Ansätze zur Verwaltungsmodernisierung verfolgt.
Im Folgenden sollen die Konzepte \glqq{}Neues Steuerungsmodell\grqq{} und \glqq{}E-Government\grqq{} kurz vorgestellt werden.

\textbf{Neues Steuerungsmodell}\\%Werner Jann Neues Steuerungsmodell, Zusammenfassung
Das Neue Steuerungsmodell ist ein Ansatz zur Verwaltungsmodernisierung, der auf den Prinzipien von Steuerung und Regulierung basiert. 
Es zielt darauf ab, die Effizienz und Wirksamkeit der Verwaltung zu erhöhen, indem klare Ziele, Verantwortlichkeiten und Überwachungsmechanismen eingeführt werden.
Das Neue Steuerungsmodell wird auch von Jann als die \glqq{}spezifisch deutsche Version\grqq{} \citep[vgl.][S. 127]{Jann2019} des \glqq{}New Public Managements\grqq{} bezeichnet.
Bei beiden Modellen geht es darum, den bürokratischen Verwaltungscharakter abzustreifen und öffentliche Verwaltungen zu Dienstleistungsunternehmen umzubauen.
In Deutschland bezieht sich das in erster Linie auf die kommunale Ebene und hat seine Ursprünge im den frühen 1990er Jahren.
Dabei finden einige Leitlinien aus der Privatwirtschaft Anwendung, wie die Leistungsorientierung, Produktorientierung oder Kundenorientierung.
Auf die Zielsetzung bezogen können laut Jann so drei Kernelemente des NSM \citep[vgl.][S. 127]{Jann2019} ausgemacht werden:

\begin{itemize}
  \item Aufbau einer dezentralen Führungs- und Organisationsstruktur
  \item Steuerung der Verwaltung durch politische Kontrakte
  \item Wettbewerb und Kundenorientierung
\end{itemize}

Voraussetzung für eine effektive Steuerung der Haushaltsmittel unter Berücksichtigung der Produktorientierung ist jedoch eine Haushaltsführung, die die zeitliche Entwicklung abbildet.
Mit der traditionellen kameralistischen Haushaltsführung der Kommunen kann in erster Linie der Ist-Zustand eines Jahres, bezogen auf Einnahmen und Ausgaben, festgestellt werden.
Durch die Umstellung auf die in der privatwirtschaft übliche bzw. verpflichtende doppelte Buchführung, die sog. Doppik, wird die Beurteilung der aktuellen Situation unter Berücksichtigung von Vermögenswerten ermöglicht.
So können insgesamt verlässlichere Entscheidungen aufgrund fundierter Informationen getroffen werden, wobei beispielsweise Kredite und zukünftige Pensionen mit einbezogen werden.
Diese Umstellung wurde jedoch erst 2003 im Rahmen des Neuen kommunalen Finanzmanagements von der Innenministerkonferenz angestoßen, wobei es keine bundeseinheitliche Regelung gibt \citep[][]{Bogumil2017}.
So nutzen im Jahr 2017 nur 7.748 von 11.927 deutschen Kommunen die Doppik, während Bayern und Thüringen auf eine erweiterte Kameralistik setzen \citep[][]{Burth2017}.
Diese zeichnet sich hauptsächlich durch eine zusätzliche Kosten- und Leistungsrechnung aus, um den Zielen des NSM gerecht zu werden.
Im Kern handelt es sich jedoch weiterhin um die klassische Kameralistik, die auf unterschiedliche Elemente der Betriebswirtschaft zurückgreift.

Die produktorientierte Steuerung in Verbindung mit einer dezentralen Organisationsstruktur soll in erster Linie die Prozesse innerhalb der Verwaltung beschleunigen.
Durch die Zuweisung eines Budgets mit Bindung an vereinbarte Ziele entfällt die Verhandlung und Genehmigung von Einzelfällen.
Das heißt, die Fachbereiche können in Eigenverantwortung effektiv und effizient handeln.
Um diesen Prozess voranzutreiben wird das Mittel des Wettbewerbs zwischen den Kommunen anhand von Kennzahlen eingeführt.
So soll eine stetige Verbesserung der Prozesse angetrieben werden.

\textbf{E-Government}\\
E-Government meint laut Bundesregierung den \glqq{}Einsatz elektronischer Informationstechnologien, damit Verwaltungsangebote für jedermann einfach, schnell und ortsunabhängig zugänglich sind\grqq{} \citep[][]{Bundesregierung2023}.
Dabei liegt der Fokus darauf, elektronische Entsprechungen bereits vorhandener Verwaltungsangebote sowie die dazu notwendigen Voraussetzungen zu schaffen.

Die Voraussetzzungen für den Einsatz von elektronischer Informationstechnologie im Rahmen von Verwaltungsdienstleistungen wurden mit dem E-Government-Gesetz (2013) geschaffen.
Zentrale Punkte dieses Gesetzes sind die Abschaffung von Schriftformerfordernissen und Erfordernissen zur persönlichen Vorsprache in verschiedenen Fachgesetzen.
In Verbindung mit der Einrichtung von De-Mail als elektronische Schriftform, sowie der Möglichkeit der elektronischen Identifizierung durch die eID-Funktion des Personalausweises, können so viele Dienstleistungen elektronisch angeboten werden \citep[][]{BMI2023}.
Durch die Verabschiedung des Onlinezugangsgesetzes (2017) wurden Bund, Länder und Kommunen zur Bereitstellung eben dieser Dienstleistungen verpflichtet.
Grundlegendes Ziel ist dabei die Errichtung eines Portalverbunds, in dem sich Bürger mit einem Nutzerkonto anmelden und anschließend alle verfügbaren Verwaltungsdienstleistungen von Bund, Ländern und Kommunen nutzen können \citep[][]{Martini2019}.

Die Kombination dieser Gesetze bildet die Anwendung von E-Government als Methode zur Verwaltungsmodernisierung in Deutschland ab und zeigt deutlich das Hauptziel von E-Government auf.
Durch die Ersetzung von Schriftformerfordernissen und persönlicher Vorsprache durch elektronische Entsprechungen entfällt der Gang zur Behörde für viele Verwaltungsdienstleistungen.
So können Prozesse effizienter gestaltet und Bürokratie abgebaut werden.
\subsection{Abgrenzung der vorgestellten Modelle}
Zur Abgrenzung der Verwaltungsmodernisierungsmodelle Neues Steuerungsmodell, E-Government und Smart Government werden diese anhand einiger Kernelemente der vorgestellten Modelle gegenübergestellt.
Dabei gibt es offensichtlich grundlegende Unterschiede zwischen NSM und den beiden \glqq{}modernen\grqq{} Methoden E-Government und Smart Government, während letztere thematisch deutlich näher zusammen liegen.

\textbf{Steuerung der Maßnahmen}\\
Die Entwicklung des NSM ging hauptsächlich von den Kommunen aus und es gab lediglich eine Empfehlung zur Einführung der Doppik als zentralen Baustein von der Innenministerkonferenz 2003.
Das führte in der Folge zu unterschiedlichen Ausgestaltungen der Länder was die Pflicht zur Einführung aber auch die Fristen betrifft \citep[][]{Mehde2019}.
Im Gegensatz dazu wird die Einführung des E-Governments durch Bundesgesetze gesteuert wodurch Länder und Kommunen zum einen befähigt, im Anschluss aber auch dazu verpflichtet werden bspw. Dienstleistungen elektronisch zur Verfügung zu stellen.
Diese Gesetze (speziell das E-Government-Gesetz) ermöglichen zusätzlich die Anwendung von Elementen aus dem Bereich Smart Government.
Speziell sei hier der Bereich Open Data genannt.

\textbf{Adressat der Maßnahmen}\\
%NSM intern
Zwar hat sich mit dem NSM der Fokus der Verwaltung auf den Bürger als Kunden gerichtet, die Konsequenzen in der Gestaltung zielen jedoch hauptsächlich auf interne Veränderungen struktureller Natur ab.
So hat die Umstellung des Haushalts und die Einführung einer dezentralen Organisationsstruktur unmittelbar keine wahrnehmbaren Auswirkungen für den Bürger.
%E-Government intern erleichterung durch Effizienz (Risiken beachten), Bürger + Wirtschaft + andere Behörden
Mit der Einführung von E-Government ändert sich das.
Zwar ist auch hier der erste Schritt eine interne Modernisierung der Prozesse, konkret die Digitalisierung, im Anschluss werden diese Prozesse jedoch zusätzlich für den Bürger geöffnet.
Dazu werden Schnittstellen zur Verfügung gestellt über die der Bürger Anträge oder Materialien wie Nachweise elektronisch an die Behörde senden kann.
Dabei steht die Verhinderung eines Medienbruchs zur Verbesserung der Effizienz und zum Abbau von Bürokratie im Vordergrund.
Neben dem Bürger profitieren zusätzlich Firmen aus der Privatwirtschaft von der neuen Art der Kommunikation mit Behörden, beispielsweise bei der Zusammenarbeit im Rahmen öffentlicher Verträge.
%Smart Government Bürger und mehr Behörden und Wirtschaft
Dieser Trend setzt sich im Smart Government fort.
Hier spielt zusätzlich die Zusammenarbeit von Behörden eine große Rolle, beispielsweise bei der Informationsbeschaffung für die Bewertung von Anträgen.
Mithilfe gemeinsamer Datenbanken und automatisierter (Teil-)Prozesse können so Vorgänge beschleunigt und Bürokratie abgebaut werden.

\textbf{Auswirkung auf Prozesse}\\
%NSM interne Umstellung, strukturelle Veränderung
Die Einführung einer dezentralen Organisationsgestaltung sowie die Umstellung auf einen produktorientierten Haushalt haben große strukturelle Auswirkungen auf die Arbeitsweise innerhalb einer Kommune.
So werden plötzlich Entscheidungen innerhalb eines Fachbereichs getroffen, was neue Herausforderungen an Führungskräfte aber auch die Mitarbeiter nach sich zieht.
Insbesondere der Wettbewerb (über Kennzahlen) mit anderen Kommunen führt im weiteren Verlauf zu Anpassungen der veränderten Prozesse.
Dabei steht stets das Ziel im Vordergrund, diese Prozesse ständig weiterzuentwickeln und effizienter zu gestalten.
%E-Government digitalisierung der Prozesse inkl. Anbieten einer Schnittstelle für den Bürger
Die Einführung von E-Government hat nicht weniger große Auswirkungen auf die Prozesse innerhalb der Verwaltung.
Jedoch beziehen sich diese Auswirkungen nicht auf grundsätzliche Änderungen von Abläufen sondern auf den technischen Aspekt.
So werden bestehende, zum Teil in Papierform vollzogene, Arbeitsabläufe digitalisiert.
Herausforderungen entstehen dabei weniger auf struktureller Ebene als vielmehr im Umgang mit entsprechender Software wie betrieblichen Informationssystemen.
%Smart Government automatisierung durch KI, Big data
Durch den Einsatz von künstlicher Intelligenz und behördenübergreifend verfügbaren Daten geht der Smart Government Ansatz noch einen Schritt weiter.
Ausdrücklich wird hier neben der Digitalisierung auch eine Optimierung der Prozesse angestrebt, wobei zunächst Routineaufgaben vereinfacht bzw. automatisiert werden sollen.
Gleichzeitig wird eine weitere Öffnung der Prozesse im Sinne des Open Governments angestrebt, sodass sich Bürger und Unternehmen selbstständig über den Fortschritt von Anträgen informieren können.




\subsection{Betrachtung der unterschiedlichen Einflüsse auf die Effizienz der Verwaltung}
Mit Effizienz ist hier die wirtschaftliche Betrachtung der verwendeten Ressourcen in Bezug auf das damit erzielte Ergebnis gemeint.
Auf die Verwaltung bezogen gibt es verschiedene Gesichtspunkte die zu einer effizienten Verwaltung beitragen können.
Im weiteren Verlauf soll der (mögliche) Einfluss der vorgestellten Methoden zur Verwaltungsmodernisierung auf einige dieser Punkte untersucht werden.

\textbf{Schlanke und effiziente Prozesse}\\
Der wohl wichtigste Faktor bei der Untersuchung der Effizienz der öffentlichen Verwaltung ist die Betrachtung der einzelnen Arbeitsabläufe.
Eine konkrete Messbarkeit für die Effizienz von Prozessen bedingt eine genaue Aufstellung der für das Erreichen eines bestimmten Ergebnis aufgewendeten Mittel.
Mit der Einführung des NSM und der Umstellung auf einen produktorientierten Haushalt kann die Voraussetzung dafür geschaffen werden.
Denn erst durch Wettbewerb und den Vergleich von Kennzahlen, die eng mit der Effizienz verbunden sind, besteht die Notwendigkeit zu effizientem Verwaltungshandeln \cite[][]{Jann2019}.
Vor allem der Aufbau einer dezentralen Organisationsstruktur und Ressourcenverantwortung führt zu kürzeren Wegen innerhalb der Organisation, da Entscheidungen nicht mehr von oberster Ebene abgesegnet werden müssen.

Die Digitalisierung der Prozesse im Zuge von E-Government kann darauf aufbauen und zusätzliche Zeitersparnisse erreichen.
Beispielsweise werden für digitale Prozesse keine Umlaufmappen benötigt, sodass der nächste Prozessschritt unmittelbar folgen kann.
Eine wichtige Voraussetzung hierbei ist jedoch, dass die Prozesse vollständig frei von Medienbrüchen sind.
Dazu sind einheitliche Schnittstellen und Formate nötig, zunächst innerhalb einer Behörde, für weitere Optimierung aber mitunter landes- oder bundesweit.
Die Optimierung der Prozesse findet hier vor allem auf technischer Ebene statt, während beim NSM strukturelle Änderungen der Organisation zu verbesserten Prozessen führt.

Durch die Automatisierung von (Teil-)Prozessen mittels künstlicher Intelligenz stellt Smart Government den Wegfall von Routineaufgaben in Aussicht.
Damit würden Mitarbeiter entlastet und könnten sich beispielsweise auf den Erlass eines Verwaltungsaktes konzentrieren statt auf das Einholen aller benötigten Information.
Zusätzlich ist die Anwendung von Errungenschaften aus dem Internet der Dinge und Dienste ein wichtiger Schritt zu verbesserten Prozessen, aber auch zu neuen Möglichkeiten \citep[][]{Lucke2016}.  
Ein Beispiel für die Anwendung von Sensorik in Verbindung mit smarter Software (neuronale Netze) nennt Schuppan.
Mithilfe dieser Technologie wäre es möglich Autobahnbrücken zu überwachen, Sanierungsbedarfe rechtzeitig zu erfassen und finanzielle Mittel dafür planen zu können \citep[][]{Schuppan2016}.
Ein anderes Beispiel ist die Warnung vor Tsunamis durch smarte Bojen \citep[][]{Lucke2016}.
Hier werden vor allem auch Prozesse möglich, die ohne die Einbindung von smarten Geräten nur schwer bis garnicht durchführbar wären.

\textbf{Kostenbetrachtung}\\


\subsection{Anteilnahme des Bürgers bei verschiedenen Modellen}
Wie bereits in Kapitel~\ref{VortBürgerbeteiligung} dargelegt, stellt die Verbesserung der Bürgerbeteiligung ein hohes Qualitätsmerkmal für die Verwaltungsmodernisierung dar.
Mit der Ausrichtung von Verwaltung als Dienstleister für den Bürger hat die Ermöglichung und Vereinfachung von Bürgerbeteiligung immer mehr an Bedeutung gewonnen.
In diesem Kapitel werden die vorgestellten Modelle zur Verwaltungsmodernisierung auf ihre Methoden zur Bürgerbeteiligung untersucht.

Das NSM schafft durch die grundsätzliche Ausrichtung auf den Bürger auch hier die Grundlage.
Zwar werden keine konkreten Maßnahmen zur Einbeziehung der Bürger vorgeschlagen, allerdings muss gerade in den 90er und 2000er Jahren zunächst ein bürgerbezogenes Bewusstsein innerhalb der Verwaltung entwickelt werden.

Die Digitalisierung ermöglicht im Rahmen von E-Government auch die Einführung elektronischer Angebote zur Bürgerbeteiligung.
Dabei erfahren vor allem traditionelle Abstimmungsvarianten und Kooperationsverfahren eine Aufwertung durch elektronische Modelle \citep[][]{Leitner2018}.
Die daraus resultierenden Angebote werden unter dem Schlagwort E-Partizipation zusammengefasst und beinhalten mehrere Intensitätsstufen, die sich in ihrer Verbindlichkeit auszeichnen.

So geht es bei der ersten Stufe um die Bereitstellung von Informationen beispielsweise durch Newsletter oder Internetauftritte. 
Daneben sind aber auch Elemente von Open Data und Open Government gemeint, wie die Bereitstellung von Daten in öffentlichen Datenbanke oder die Übertragung von parlamentarischen Sitzungen.

In der zweiten Stufe (Konsultation) wird der Bürger mit in die Kommunikation einbezogen, so sind digitale Diskussionsforen oder ein Beschwerdemanagement vorstellbar bei dem beispielsweise über eine mobile App Schlaglöcher, Graffitis o.Ä. gemeldet werden können.
Ein konkretes Beispiel ist das Formular \glqq{}Mängelmeldung\grqq{} der Stadt Gladbeck\footnote{verfügbar unter\\ \url{https://www.gladbeck.de/Rathaus_Politik/Rathaus/Online-Services/_Maengelmeldung/_Neuer_Vorgang.asp}}, über das eine genaue Beschreibung inklusive Bildern möglich ist.

Die dritte Stufe unterscheidet zwischen Angeboten zur Kooperation sowie Mitentscheidung.
Bei der Kooperation geht es zum Beispiel darum, Bürger bei der Erstellung von Beschlussvorlagen oder bei der Stadtentwicklungsplanung aktiv mit einzubeziehen.
Das kann durch die Möglichkeit der Mitarbeit an gemeinsamen Dokumenten geschehen oder durch Einreichung von Ideen und Bedenken über das Internet. 
Bei der Mitentscheidung als intensivste Stufe können, unter Beachtung rechtlicher Voraussetzungen, durch E"~Voting, Online-Umfragen oder E"~Petitionen Grundlagen für Beschlüsse mitbestimmt werden \citep[vgl.][]{Leitner2018}.

Die Hauptfaktoren bei allen Angeboten zur E-Partizipation sind eine möglichst niedrige Beteiligungshürde bei gleichzeitiger sicherer Identifikation.
Entsprechende Instrumente müssen im Zuge der Verwaltungsmodernisierung durch E-Government bereitgestellt werden (vgl. Kapitel~\ref{Ueberblick}).

Smart Government zielt in diesem Zusammenhang auf die konsequente Weiterentwicklung von E-Partizipation unter verstärkter Einbindung von Open Data und Open Government Strategien ab.






%Smart Government: Portal für Services, Zugriff auf Infos der Verwaltung (Transparenz führt zu Vertrauen), Kommunikation, Datenbanken (Open Data, Government)
%Bürgerbeteiligung als Qualitätsmerkmal für Verewaltungsmodernisierung (ausgeprägt durch online Volksabstimmungen)