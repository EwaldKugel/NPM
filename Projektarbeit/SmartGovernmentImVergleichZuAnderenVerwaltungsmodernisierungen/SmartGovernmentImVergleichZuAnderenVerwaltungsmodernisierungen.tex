\section{Smart Government im Vergleich zu anderen Verwaltungsmodernisierungen}
In diesem Kapitel werden zunächst die zu Beginn kurz vorgestellten Modelle zur Verwaltungsmodernisierung voneinander abgegrenzt.
Dazu werden die Ziele und konkreten Umsetzungen zur Erreichung der jeweiligen Ziele miteinander verglichen.
Anschließend wird der (mögliche) Einfluss dieser Modelle auf zwei Kernpunkte der öffentlichen Verwaltung diskutiert.
Dabei wird zunächst die Effizienz innerhalb der Verwaltung betrachtet, bevor auf die Kommunikation mit und die Beteiligung von Bürgern an der Verwaltung behandelt wird.
\subsection{Überblick über andere Verwaltungsmodernisierungsarten}
In Deutschland gibt es neben dem Smart Government noch weitere Ansätze zur Verwaltungsmodernisierung, die ebenfalls zu einer effizienteren und bürgernahen Verwaltung beitragen können. 
Diese Ansätze werden im Folgenden vorgestellt.

\begin{itemize}
 \item Neues Steuerungsmodell (NSM):\\
Das Neue Steuerungsmodell (NSM) ist ein Ansatz zur Verwaltungsmodernisierung, der auf den Prinzipien von Steuerung und Regulierung basiert. 
Es zielt darauf ab, die Effizienz und Wirksamkeit der Verwaltung zu erhöhen, indem klare Ziele, Verantwortlichkeiten und Überwachungsmechanismen eingeführt werden.
Dabei basiert es auf die Einführung einer dezentralen Führungs- und Organisationssstruktur, Outputsteuerung, d. h. Instrumenten zur Steuerung der Verwaltung von der Leistungsseite her, sowie der Aktivierung dieser neuen Struktur durch Wettbewerb und Kundenorientierung \citep[Vgl.][S.131]{Veit2019}
Insgesamt zielt das NSM darauf ab, die Verwaltungsprozesse zu standardisieren und eine effizientere Verwaltung zu ermöglichen.
\item Serviceorientierte Verwaltung (SOV):\\
Die Serviceorientierte Verwaltung (SOV) ist ein Ansatz zur Verwaltungsmodernisierung, der auf der Serviceorientierung in der Verwaltung basiert. 
Es zielt darauf ab, die Verwaltungsdienstleistungen auf die Bedürfnisse der Bürger auszurichten und eine bürgernahe Verwaltung zu ermöglichen.
``Es wird deutlich, dass mittels einer klar strukturierten und auf die Beteiligung der Kunden abstellenden intensiven Beschäftigung mit dem jeweiligen Leistungsangebot die Qualität der von Kommunen produzierten Dienstleistungen gesteigert werden könnte'' \citep[][S. 35]{Vogelgesang2016}.  
Die SOV kann die Zufriedenheit der Bürger mit der Verwaltung erhöhen und eine bürgernahe Verwaltung fördern.
\item Verwaltung 4.0:\\
Verwaltung 4.0 basiert auf dem Gedanken der Industrie 4.0, also der so genannten vierten industriellen Revolution durch Digitalisierung.
Grob gesagt zielt Verwaltung 4.0 darauf ab, möglichst alle Prozesse und Dienstleitungen der Verwaltung zu digitalisieren.
Dabei ist der produktionsorientierte Ansatz, wie z.B. Dienstleistungen oder Angeboten von Verwaltung für den Bürger, im Mittelpunkt \citep[Vgl.][S. 27-28]{Schuppan2016}
\end{itemize}

Zusammenfassend gibt es neben dem Smart Government noch weitere Ansätze zur Verwaltungsmodernisierung in Deutschland, die ebenfalls zu einer effizienteren und bürgernahen Verwaltung beitragen können. 
Diese Ansätze umfassen unter anderem das Neue Steuerungsmodell, die Serviceorientierte Verwaltung und Verwaltung 4.0.
Dabei ist allerdings zu beachten, dass die verschiedenen Verwaltungsmodernisierungsarten teilweise inhaltlich übereinstimmen und nicht ganz klar abzugrenzen sind.
All diese Arten setzen mehr oder weniger auf Technologien um die Verwaltung zu Modernisieren.
\subsection{Kostenvergleich}
Eine der wichtigsten Überlegungen bei der Implementierung neuer Technologien in Verwaltungen ist die Kosteneffizienz.
In diesem Unterkapitel werden wir uns auf den Vergleich der Kosten zwischen Smart Government und anderen Verwaltungsmodernisierungsarten in Deutschland konzentrieren.
\par
Zu den Kosten, die bei der Implementierung von Smart Government berücksichtigt werden müssen, gehören sowohl die Anschaffungskosten für die notwendige Technologie als auch die laufenden Kosten für Wartung und Betrieb.
Außerdem müssen eventuelle Umstellungskosten auf die neue Technologie ebenfalls berücksichtigt werden.
Andere Verwaltungsmodernisierungsarten, wie beispielsweise die Automatisierung von Prozessen oder die Digitalisierung von Dokumenten, haben ebenfalls Kosten, die es zu berücksichtigen gilt.
\par
Um einen Vergleich zu ermöglichen, müssen jedoch die Vorteile, die jede Verwaltungsmodernisierung bietet, ebenfalls berücksichtigt werden.
So kann Smart Government beispielsweise durch die Integration von Daten und Prozessen zu einer effizienteren Verwaltung beitragen, was wiederum zu Einsparungen in der Verwaltung führen kann.
Andere Verwaltungsmodernisierungsarten können ebenfalls zu Einsparungen führen, beispielsweise durch die Reduzierung von Papier- und Druckkosten.
\par
Die Analyse der Kosten ist jedoch nicht einfach, da es schwierig ist, die Vorteile jeder Verwaltungsmodernisierung genau zu quantifizieren.
Trotzdem ist es wichtig, einen Vergleich anzustellen, um die besten Optionen für die Verwaltung zu identifizieren.
In diesem Unterkapitel werden wir daher eine kritische Bewertung der Kosten von Smart Government im Vergleich zu anderen Verwaltungsmodernisierungsarten vornehmen.
\subsection{Betrachtung der unterschiedlichen Einflüsse auf die Effizienz der Verwaltung}
Mit Effizienz ist hier die wirtschaftliche Betrachtung der verwendeten Ressourcen in Bezug auf das damit erzielte Ergebnis gemeint.
Auf die Verwaltung bezogen gibt es verschiedene Gesichtspunkte die zu einer effizienten Verwaltung beitragen können.
Im weiteren Verlauf soll der (mögliche) Einfluss der vorgestellten Methoden zur Verwaltungsmodernisierung auf die Einführung schlanker und effizienter Prozesse untersucht werden.

Der wohl wichtigste Faktor bei der Untersuchung der Effizienz der öffentlichen Verwaltung ist die Betrachtung der einzelnen Arbeitsabläufe.
Eine konkrete Messbarkeit für die Effizienz von Prozessen bedingt eine genaue Aufstellung der für das Erreichen eines bestimmten Ergebnis aufgewendeten Mittel.
Mit der Einführung des NSM und der Umstellung auf einen produktorientierten Haushalt kann die Voraussetzung dafür geschaffen werden.
Denn erst durch Wettbewerb und den Vergleich von Kennzahlen, die eng mit der Effizienz verbunden sind, besteht die Notwendigkeit zu effizientem Verwaltungshandeln \cite[][]{Jann2019}.
Vor allem der Aufbau einer dezentralen Organisationsstruktur und Ressourcenverantwortung führt zu kürzeren Wegen innerhalb der Organisation, da viele Entscheidungen nicht mehr von oberster Ebene abgesegnet werden müssen.

Die Digitalisierung der Prozesse im Zuge von E-Government kann darauf aufbauen und zusätzliche Zeitersparnisse erreichen.
Beispielsweise werden für digitale Prozesse keine Umlaufmappen benötigt, sodass der nächste Prozessschritt unmittelbar folgen kann.
Eine wichtige Voraussetzung hierbei ist jedoch, dass die Prozesse vollständig frei von Medienbrüchen sind.
Dazu sind einheitliche Schnittstellen und Formate nötig, zunächst innerhalb einer Behörde, für weitere Optimierung aber mitunter landes- oder bundesweit.

Durch die Öffnung der Prozesse und gleichzeitige Einbindung des Bürgers in den Prozessablauf können weitere Verbesserungen erreicht werden.
Dabei helfen standardisierte Eingabemasken sowie die Übermittlung erforderlicher Unterlagen in digitaler Form.
Ohne die Notwendigkeit der persönlichen Vorsprache wird dem Sachbearbeiter die Arbeit abgenommen, die (E-)Akte zu erstellen und eventuell Dokumente zu kopieren oder zu digitalisieren.
Dies passiert bei der elektronischen Antragstellung durch geführte Formulare, in denen der Bürger die erforderlichen Daten hinterlegt und Dokumente hochladen kann.
Im Idealfall erhält der Sachbearbeiter eine vollständige Akte und kann den Sachverhalt direkt beurteilen, zumindest aber wird eine Akte mit Kerndaten angelegt und es müssen eventuell benötigte Unterlagen nachgefordert werden.

Durch die Automatisierung von (Teil-)Prozessen mittels künstlicher Intelligenz stellt Smart Government den Wegfall von Routineaufgaben in Aussicht.
Damit würden Mitarbeiter entlastet und könnten sich beispielsweise auf den Erlass eines Verwaltungsaktes konzentrieren statt auf das Einholen aller benötigten Informationen.
Im oben konstruierten Beispiel würde der Bürger in einem entsprechenden elektronischen Formular einwilligen, dass bestimmte bei anderen Behörden oder Institutionen hinterlegte Daten automatisch eingeholt und in den Prozess miteinbezogen werden.
Dies wäre eine weitere Entlastung für alle Prozessteilnehmer, inklusive dem Bürger.
So könnten beispielsweise zusätzliche Anträge entfallen die für die Einholung von Informationen notwendig sein können, da diese im Hintergrund durch die Verwendung gemeinsamer Datenbanken bereits zur Verfügung stehen.
Beispiele dafür gibt es bereits in anderen Ländern, siehe Kapitel~\ref{SGEffizienz}.

Zusätzlich ist die Anwendung von Errungenschaften aus dem Internet der Dinge und Dienste ein wichtiger Schritt zu verbesserten Prozessen, aber auch zu neuen Möglichkeiten \citep[][]{Lucke2016}.  
Ein Beispiel für die Anwendung von Sensorik in Verbindung mit smarter Software (neuronale Netze) nennt Schuppan.
Mithilfe dieser Technologie wäre es möglich Autobahnbrücken zu überwachen, Sanierungsbedarfe rechtzeitig zu erfassen und finanzielle Mittel dafür planen zu können \citep[][]{Schuppan2016}.
Ein anderes Beispiel ist die Warnung vor Tsunamis durch smarte Bojen \citep[][]{Lucke2016}.
Hier werden vor allem auch Prozesse möglich, die ohne die Einbindung von smarten Geräten nur schwer bis garnicht durchführbar wären.

Zusammengefasst lässt sich festhalten, dass die Optimierung der Prozesse beim E-Government und Smart Government vor allem auf technischer Ebene stattfindet, während beim NSM strukturelle Änderungen der Organisation zu verbesserten Prozessen führt.

\input{SmartGovernmentImVergleichZuAnderenVerwaltungsmodernisierungen/AnteilnahmeDesBürgersBeiVerschiedenenModellen}