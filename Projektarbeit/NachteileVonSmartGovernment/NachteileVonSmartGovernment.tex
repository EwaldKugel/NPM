\section{Risiken von Smart Government}
Smart Government birgt neben seinen Chancen, die Verwaltung zu verbessern allerdings auch einige Risiken.
In diesem Abschnitt werden die Risiken aufgezeigt, die die Verwendung von zahlreichen Innovationen in kurzer Zeit auslösen können.
Von Lucke spricht in diesem Kontext oft von disruptiven Wirkungen, die \glqq{}Herausforderungen für die eigene Wertschöpfung, Geschäftsmodellen, nachgelagerten Dienstleistungen und die Arbeitsorganisation bedeuten\grqq{} \citep[][S.172	]{von_Lucke_2016}.
Das bedeutet, dass es durchaus negative Einflüsse auf die Umsetzung einer moderneren Verwaltung durch Smart Government gibt. 
\par
Dabei sollten die Handlungsfelder, in denen autonome Technologien im Rahmen des Smart Governments eingesetzt werden, stark überdacht werden.
So ist in den meisten Fällen davon abzuraten autonome Computersysteme bei wichtigen Entscheidungen, zum Beispiel der Entscheidung über Haftstrafen, zu benutzen.
Laut Tonn und Stiefel könnte es sogar einen Verlust an Menschlichkeit geben, sollten solche Entscheidungen nicht mehr durch den Menschen getroffen werden \citep[Vgl.][S. 322]{Tonn2012}.
\subsection{Kosten der Einführung von Smart Government}
Der Kostenfaktor ist einer der größten Nachteile bei der Einführung intelligenter Verwaltung in Deutschland.
Der Einsatz von Technologie, um die Verwaltung zu verbessern, erfordert eine signifikante Investition in die notwendige Infrastruktur, die sich aufgrund der Komplexität und der erforderlichen technischen Fähigkeiten als sehr teuer erweisen kann.
Die Kosten für die Entwicklung der Infrastruktur und die Implementierung eines digitalen Systems, das alle staatlichen Dienste abdeckt, können zu erheblichen finanziellen Belastungen für die Verwaltung führen.
Weiterhin kann das Vertrauen der Bürger unter den immensen Kosten zur Umsetzung leiden, so wurde in einem Gutachten für den nationale Normenkontrollrat 2015 festgestellt, dass etwa 62\% Angst haben, dass die \glqq{}ohnehin defizitären Haushalte mit unverhältnismäßig hohen Kosten belastet\grqq{} werden \citep[][S. 16]{NationalenNormenkontrollrat2015}.
\par
Darüber hinaus können die Kosten ansteigen, wenn das System erweitert werden muss, um neue Technologien zu unterstützen oder die Sicherheit der Systeme zu verbessern.
Um die Kosten für die Implementierung intelligenter Verwaltungsmaßnahmen zu senken, muss die Bundesregierung Finanzhilfen von Unternehmen und Organisationen erhalten, die in die Entwicklung von Technologien und die Implementierung von Systemen investieren \citep[][S. 51]{NationalenNormenkontrollrat2015}.
\par
Weiterhin müssen Investitionen in die Schulung von Personen getätigt werden, die das neue System verwalten und unterstützen, wodurch die Kosten weiter erhöht werden.
Auch die Kosten für den Betrieb des Systems müssen berücksichtigt werden, was zu weiteren erheblichen Kosten für die Verwaltung führen kann \citep[][S. 52, 88]{NationalenNormenkontrollrat2015}.
\clearpage{}
Der Kostenfaktor ist ein ernstzunehmender Nachteil bei der Einführung von Smart Government.
Es ist notwendig, dass die Verwaltung Finanzhilfen von Organisationen erhält, um die Kosten der Implementierung des Systems zu senken und um die Kosten für die Schulung der Personen, die das System verwalten, zu decken.
Allerdings gibt es durchaus auch Potentiale zur Kosteneinsparungen bei der Digitalisierung von papierbasierten Prozessen, wie zum Beispiel Einsparmöglichkeiten von Papierverbrauch, Energiekosten und notwendigen Eingriffen durch Verwaltungsmitarbeiter \citep[Vgl.][S.179]{von_Lucke_2016}.
\subsection{Potentielle Möglichkeiten von Datenmissbrauch}
Der Missbrauch von Daten ist ein ernstes Problem, wenn es um die Einführung von Smart Government in Deutschland geht.
Datenmissbrauch kann verschiedene Formen annehmen, beispielsweise, dass sensible Daten von Bürgern missbraucht werden, um persönliche Vorteile zu erhalten oder in die Handelskontrolle einzugreifen.
Diese Daten, oftmals Sensordaten die im Zuge der Digitalisierung erhoben werden, lassen sich benutzen um das Verhalten von Personen weltweit durch Raum und Zeit zu verfolgen.
Rückschlüsse auf das Verhalten durch Auswertung dieser Daten von Bürgern sind sensible Themen, die das Vertrauen des Bürgers bei Missbrauch deutlich schwächen kann \citep[Vgl.][S.179]{von_Lucke_2016}.
Dies kann die Privatsphäre der Bürger gefährden und die Sicherheit des Staates in Gefahr bringen.
\par
Dabei ist es essenziell, dass der Staat sich die Frage der Regulierung des Umgangs mit solcher sensiblen Daten stellt.
Es ist notwendig Grenzen bei Datensammlung und Datenauswertung  per Gesetz zu setzen, sodass es klare Regelgungen beispielweise bei der Verwendung von Daten aus intelligenten Systemen wie Smart Watches, Smart Tv's oder auch Smart Pad's.
Natürlich können diese Daten helfen die Verwaltung zu verbessern und dem Bürger helfende Systeme bereitzustellen (siehe Beispiel \ref{LAVerkehr}), allerdings ist die Grenze zum Missbrauch solcher Daten sehr dünn \citep[Vgl.][S. 39]{Lucke2016}.
\par
Darüber hinaus kann dieser Datenmissbrauch zu ebendiesem Rückgang der Vertrauenswürdigkeit der Verwaltung führen.
Wenn die Bürger erkennen, dass die Regierung ihre Daten missbraucht, verlieren sie Vertrauen in das System und können sich nicht mehr darauf verlassen, dass ihre Daten sicher sind.
Dies kann auch dazu führen, dass die Menschen nicht mehr bereit sind, mit der Regierung zu kooperieren, sodass der eigentliche Vorteil der stärkeren Bürgerbeteiligung zunichte gemacht wird (siehe Abschnitt \ref{VortBürgerbeteiligung}).
\par
Datenmissbrauch ist ein ernstes Problem, wenn es um die Einführung von Smart Government in Deutschland geht.
Daher ist es wichtig, dass die Regierung ein effektives System zur Unterbindung solcher Verhaltensweisen entwickelt.
Darüber hinaus sollte die Regierung Programme einführen, die den Bürgern Informationen über die verschiedenen Rechte und Pflichten im Zusammenhang mit der Verwendung ihrer persönlichen Daten geben.
\input{NachteileVonSmartGovernment/BetrachtungVonMöglichenSicherheitslückenBeiDerNutzungVonSmartGovernment}