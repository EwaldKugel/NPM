\subsection{Potentielle Möglichkeiten von Datenmissbrauch}
Der Missbrauch von Daten ist ein ernstes Problem, wenn es um die Einführung von Smart Government in Deutschland geht.
Datenmissbrauch kann verschiedene Formen annehmen, beispielsweise, dass sensible Daten von Bürgern missbraucht werden, um persönliche Vorteile zu erhalten oder in die Handelskontrolle einzugreifen.
Diese Daten, oftmals Sensordaten die im Zuge der Digitalisierung erhoben werden, lassen sich benutzen um das Verhalten von Personen weltweit durch Raum und Zeit zu verfolgen.
Rückschlüsse auf das Verhalten durch Auswertung dieser Daten von Bürgern sind sensible Themen, die das Vertrauen des Bürgers bei Missbrauch deutlich schwächen kann \citep[Vgl.][S.179]{von_Lucke_2016}.
Dies kann die Privatsphäre der Bürger gefährden und die Sicherheit des Staates in Gefahr bringen.
\par
Dabei ist es essenziell, dass der Staat sich die Frage der Regulierung des Umgangs mit solcher sensiblen Daten stellt.
Es ist notwendig, Grenzen bei Datensammlung und Datenauswertung  per Gesetz zu setzen, sodass es klare Regelungen beispielweise bei der Verwendung von Daten aus intelligenten Systemen wie Smart Watches, Smart Tv's oder auch Smart Pad's gibt.
Natürlich können diese Daten helfen die Verwaltung zu verbessern und dem Bürger helfende Systeme bereitzustellen (siehe Beispiel~\ref{LAVerkehr}), allerdings ist die Grenze zum Missbrauch solcher Daten sehr dünn \citep[Vgl.][S.~39]{Lucke2016}.
\par
Darüber hinaus kann dieser Datenmissbrauch zu ebendiesem Rückgang der Vertrauenswürdigkeit der Verwaltung führen.
Wenn die Bürger erkennen, dass die Regierung ihre Daten missbraucht, verlieren sie Vertrauen in das System und können sich nicht mehr darauf verlassen, dass ihre Daten sicher sind.
Dies kann auch dazu führen, dass die Menschen nicht mehr bereit sind, mit der Regierung zu kooperieren, sodass der eigentliche Vorteil der stärkeren Bürgerbeteiligung zunichte gemacht wird (siehe Abschnitt \ref{VortBürgerbeteiligung}).
\par
Datenmissbrauch ist ein ernstes Problem, wenn es um die Einführung von Smart Government in Deutschland geht.
Daher ist es wichtig, dass die Regierung ein effektives System zur Unterbindung solcher Verhaltensweisen entwickelt.
Darüber hinaus sollte die Regierung Programme einführen, die den Bürgern Informationen über die verschiedenen Rechte und Pflichten im Zusammenhang mit der Verwendung ihrer persönlichen Daten geben.