\subsection{Betrachtung von möglichen Sicherheitslücken bei der Nutzung von Smart Government}
Eine Einführung von Smart Government in Deutschland kann zu einer Reihe von Sicherheitslücken führen.
Folgend werden einige potentielle Risiken beschrieben, die bei der Nutzung von Smart Government drohen.
\par
Die Entwicklung und das Betreiben von mehreren neuen Anwendungen und Systeme, kann über das Netzwerk anfällig für Cyberangriffe sein.
Da mehr Informationen über das Netzwerk geteilt werden, können Cyberangriffe leichter durchgeführt werden.
Dies kann zu einem Verlust von Daten führen, der die gesamte Regierungsinfrastruktur beeinträchtigen kann.
Weiterhin muss der Staat aufpassen nicht gegen die freiheitlich-demokratische Grundordnung zu verstoßen, da die Bürger durch eine Vielzahl an Rechten geschützt sind \citep[Vgl.][S. 179]{von_Lucke_2016}.
\par
Weiterhin können diese Systeme anfällig für Datenschutzverletzungen sein.
Da Systeme immer komplexer werden, wird der Bedarf an einer sicheren Systemarchitektur immer größer.
Wenn Systeme nicht ordnungsgemäß gesichert sind, können Benutzer Zugriff auf sensible Daten erhalten, was zu schwerwiegenden Folgen für die Datensicherheit führen kann \citep[Vgl.][]{von_Lucke_2016}.
\par
Aber auch Menschen stellen eine potenzielle Sicherheitslücke dar.
Eine starke Passwortverwaltung wird benötigt, um sicherzustellen, dass Benutzer nicht unerlaubten Zugriff auf das Netzwerk erhalten.
Außerdem muss eine starke Anwendungssicherheit implementiert werden, um sicherzustellen, dass Benutzer nur auf die Anwendungen zugreifen können, für die sie berechtigt sind.
Zur Eindämmung dieses Risikos können Sicherheitsrichtlinien zur Passwortvorgabe oder ähnlichen erstellt werden.
\par
Insgesamt besteht bei der Einführung von Smart Government in Deutschland ein hohes Risiko für Datenschutzverletzungen.
Es ist wichtig, dass Regierungen ein starkes Netzwerk, Systeme und Anwendungssicherheit implementieren, um sicherzustellen, dass die Datensicherheit gewährleistet ist.
%Mandantensicher