\subsection{Überblick über Verwaltungsmodernisierungsarten}
Im Gegensatz zu den im folgenden beschriebenen Verwaltungsmodernisierungsarten beinhaltet Smart Government eine Reihe von Technologien, die es der Verwaltung ermöglichen, effizienter, effektiver und bürgernaher zu arbeiten. 
Dazu gehören Datenanalysen, Künstliche Intelligenz, Sensorik und Big Data-Analysen, die es der Verwaltung ermöglichen, die Leistung zu messen und zu verbessern.
\par
Abgrenzend dazu beinhalten andere Verwaltungsmodernisierungen in Deutschland einen Prozess, bei dem die Verwaltung effizienter, flexibler und besser auf die Anforderungen der Bürger und Unternehmen angepasst wird. 
Dieser Prozess wird durch eine Vielzahl von Maßnahmen unterstützt, die sich auf verschiedene Aspekte konzentrieren, einschließlich der Organisation der Verwaltung, der Digitalisierung und der Nutzung von Technologien, um den Service zu verbessern. 
Zu diesen Maßnahmen zählen unter anderem die Reform der Verwaltungsstrukturen, die Einführung von IT-gestützten Verwaltungsprozessen, die Automatisierung von Verfahren und die Einführung von Online-Services.
Die Folgenen Verwaltungsmodernisierungsansätze werden dabei in dieser Arbeit dem Ansatz des Smart Governments gegenübergestellt.
\par
Eine dieser Arten ist die Digitalisierung., welche sich auf den Einsatz digitaler Technologien bezieht, um Prozesse zu vereinfachen und zu beschleunigen. 
Dies kann z.B. die Einführung einer digitalen Akte oder eines Online-Dienstes zur Verwaltung von Anträgen und Anfragen beinhalten.
\par
Eine weitere Art der Verwaltungsmodernisierung bildet die Organisationsrationalisierung.
Sie bezieht sich auf Maßnahmen, die auf eine effiziente Organisation der Verwaltung abzielen.
Beispiele hierfür sind die Einführung einer Verwaltungsstruktur, die Optimierung der Arbeitsabläufe und die Einführung von modernen und effizienteren Verwaltungsprozessen.
\par
Weiterhin fokussiert sich die Reform der Finanzverwaltung auf die Einführung neuer Verfahren und Instrumente, um die Steuereinnahmen zu erhöhen und die Effizienz der Steuerverwaltung zu steigern. 
Dazu gehören z.B. die Einführung eines elektronischen Steuerabfuhrsystems, die automatische Abwicklung von Steuererklärungen und die Modernisierung des Steuergesetzes.
\par
Die grundlegende Art der Verwaltungsmodernisierung bildet das E-Government, welche sich auf die Nutzung digitaler Technologien bezieht, um die Bürger besser zu bedienen. 
Dazu zählen z.B. der Aufbau von Online-Diensten, die Veröffentlichung von Informationen im Internet, die Schaffung einer E-Government-Plattform und die Einführung von Online-Steuerabfuhrsystemen.
\par
Der Bürokratieabbau wiederum zielt darauf ab, die Anzahl und Komplexität der behördlichen Verfahren zu verringern. 
Dazu gehören die Vereinfachung von Antragsformularen, die Entbürokratisierung von Verfahren und die Einführung von elektronischen Verwaltungsverfahren.
\par
Diese Arten der Verwaltungsmodernisierung stimmen teils in ihren Inhalten überein, bilden allerdings in sich geschlossen einen eigenen Ansatz zur Modernisierung der Verwaltungsprozesse.
In der Betrachtung des Smart Government sind diese Ansätze allerdings zu berücksichtigen und eventuell mit einzubinden.