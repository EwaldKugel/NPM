\subsection{Definition Smart Government}
Smart Government ist eine der neuesten Entwicklungen im Bereich der Verwaltungsmodernisierung in Deutschland. 
Es handelt sich dabei um eine Bewegung, die darauf abzielt, die öffentliche Verwaltung zu digitalisieren und effizienter zu gestalten. 
Smart Government verbindet dabei die Vorteile digitaler Technologien mit der Professionalität der öffentlichen Verwaltung. 
Es ermöglicht unter anderem eine schnellere Reaktion auf Bürgerwünsche, eine bessere Vernetzung und ein höheres Maß an Transparenz. 
\par
Im Vergleich zu anderen Formen der Verwaltungsmodernisierung in Deutschland hat Smart Government einige einzigartige Eigenschaften. 
Zunächst einmal konzentriert es sich auf die Digitalisierung der Verwaltung, anstatt auf das Management von Veränderungen. 
Es ist darauf ausgerichtet, die Verwaltung effizienter und effektiver zu machen, indem die Verwaltungsabläufe digitalisiert werden und die Kommunikation zwischen den Bürgern und der Verwaltung vereinfacht wird. 
Weiterhin ermöglicht Smart Government die Entwicklung von Anwendungen, die die Kommunikation zwischen Bürgern und der Verwaltung erleichtern und den Zugang zu öffentlichen Dienstleistungen vereinfachen. 
\par
Außerdem ermöglicht Smart Government es, dass die Verwaltung einfacher und schneller auf die Bedürfnisse der Bürger reagiert. 
Dadurch können Verwaltungsabläufe optimiert und Prozesse verbessert werden. 
Unter anderem werden dabei Datenbanken entwickelt, die die Verwaltung und die Bürger verbinden. 
Dadurch können Bürger leicht auf Informationen zu öffentlichen Dienstleistungen zugreifen, die von der Verwaltung bereitgestellt werden.
\par
In dieser Arbeit werden die Vor -und Nachteile der Einführung von Smart Government diskutiert.
Dabei wird der Vergleich mit anderen Verwaltungsmodernisierungsarten gezogen, bei dem Eigenschaften wie Transparenz, Effizienz und anderer wichtiger Faktoren für die Modernisierung miteinander verglichen werden.
