\subsection{Definition Smart Government}
Smart Government ist eine der neuesten Entwicklungen im Bereich der Verwaltungsmodernisierung in Deutschland. 
Es handelt sich dabei um eine Bewegung, die darauf abzielt, die öffentliche Verwaltung zu digitalisieren und effizienter zu gestalten. 
Smart Government soll die Verwaltung mit Hilfe intelligenter Informations- und Kommunikationstechniken verbessern \citep[][S.178]{von_Lucke_2016}.
Es ermöglicht unter anderem eine schnellere Reaktion auf Bürgerwünsche, eine bessere Vernetzung und ein höheres Maß an Transparenz \citep[Vgl][S.87]{Kersting2017}.
\par
Smart Government schließt dabei die Leistungen und Ansätze von E-Government und Open Government, inklusive Big Data und Open Data mit ein.
Zunächst einmal konzentriert es sich auf die Digitalisierung der Verwaltung, anstatt auf das Management von Veränderungen wie zum Beispiel beim Ansatz des neuen Steuerungsmodell. 
Es ist darauf ausgerichtet, die Verwaltung effizienter und effektiver zu gestalten, indem die Verwaltungsabläufe digitalisiert werden und die Kommunikation zwischen den Bürgern und der Verwaltung vereinfacht wird \citep[Vgl.][]{von_Lucke_2016}.
Weiterhin ermöglicht Smart Government die Entwicklung von Anwendungen, die die Kommunikation zwischen Bürgern und der Verwaltung erleichtern und den Zugang zu öffentlichen Dienstleistungen vereinfachen. 
Ein Beispiel dafür ist die Echtzeitüberwachung und Einsicht von Fischereikontrollen und Fangquoten mit Hilfe einer Smart Government Anwendung.
Dabei werden in Echtzeit mittels Satellitendaten oder Zugriff auf Daten von elektronischen Logbüchern Risiko- und Verdachtsfälle identifiziert \citep[][]{LandwirtschaftundErnaehrung2023}. 
\par
Ein Ziel von Smart Government ist dabei die vollständige Automatisierung von Dienstleistungen an den Bürger, sodass diese ohne weitere Bearbeitung eines Sachbearbeiters abgeschlossen werden können.
Unter anderem werden dabei Datenbanken entwickelt, die die Verwaltung und die Bürger verbinden \citep[Vgl.][]{von_Lucke_2016}. 
Dadurch können Bürger leicht auf Informationen zu öffentlichen Dienstleistungen zugreifen, die von der Verwaltung bereitgestellt werden.
\glqq{}Smart Government kann zur Steigerung der Lebensqualität, zur Verbesserung der Standortqualität und zur Stärkung der Bürgerorientierung beitragen\grqq{} \citep[][]{von_Lucke_2016}. 
\par
In dieser Arbeit werden die Chancen und Risiken der Einführung von Smart Government diskutiert.
Dabei wird der Vergleich mit anderen Verwaltungsmodernisierungsarten gezogen, bei dem Eigenschaften wie Transparenz, Effizienz und anderer wichtiger Faktoren für die Modernisierung miteinander verglichen werden.
Besonders die grundlegenden Ansätze und Ziele werden betrachtet, wobei die Gemeinsamkeiten und deren kombinierte Anwendung zur Modernisierung der Verwaltung eingesetzt werden können.

