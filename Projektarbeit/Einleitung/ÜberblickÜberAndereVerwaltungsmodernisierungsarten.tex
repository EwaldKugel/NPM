\subsection{Überblick über andere Verwaltungsmodernisierungsarten}
In Deutschland gibt es neben dem Smart Government noch weitere Ansätze zur Verwaltungsmodernisierung, die ebenfalls zu einer effizienteren und bürgernahen Verwaltung beitragen können. 
Einige wichtige dieser Ansätze werden im Folgenden vorgestellt.

\begin{itemize}
 \item Neues Steuerungsmodell (NSM):\\
Das Neue Steuerungsmodell (NSM) ist ein Ansatz zur Verwaltungsmodernisierung, der auf den Prinzipien von Steuerung und Regulierung basiert. 
Es zielt darauf ab, die Effizienz und Wirksamkeit der Verwaltung zu erhöhen, indem klare Ziele, Verantwortlichkeiten und Überwachungsmechanismen eingeführt werden.
Dabei basiert es auf die Einführung einer dezentralen Führungs- und Organisationssstruktur, Outputsteuerung, d. h. Instrumenten zur Steuerung der Verwaltung von der Leistungsseite her, sowie der Aktivierung dieser neuen Struktur durch Wettbewerb und Kundenorientierung \citep[Vgl.][S.131]{Veit2019}
Insgesamt zielt das NSM darauf ab, die Verwaltungsprozesse zu standardisieren und eine effizientere Verwaltung zu ermöglichen.
% \item Serviceorientierte Verwaltung:\\
% Die Serviceorientierte Verwaltung ist ein Ansatz zur Verwaltungsmodernisierung, der den Fokus der Modernisierung auf die Verbesserung und Weiterentwicklung des Bürgerservices setzt. 
% Es zielt darauf ab, die Verwaltungsdienstleistungen auf die Bedürfnisse der Bürger auszurichten und eine bürgernahe Verwaltung zu ermöglichen.
% ``Es wird deutlich, dass mittels einer klar strukturierten und auf die Beteiligung der Kunden abstellenden intensiven Beschäftigung mit dem jeweiligen Leistungsangebot die Qualität der von Kommunen produzierten Dienstleistungen gesteigert werden könnte'' \citep[][S. 35]{Vogelgesang2016}.  
% Die Serviceorientierte Verwaltung kann also die Zufriedenheit der Bürger mit der Verwaltung erhöhen und eine bürgernahe Verwaltung fördern.
\item Verwaltung 4.0:\\
Verwaltung 4.0 basiert auf dem Gedanken der Industrie 4.0, also der so genannten vierten industriellen Revolution durch Digitalisierung.
Grob gesagt zielt Verwaltung 4.0 darauf ab, möglichst alle Prozesse und Dienstleitungen der Verwaltung zu digitalisieren.
Dabei ist der produktionsorientierte Ansatz, wie z.B. Dienstleistungen oder Angeboten von Verwaltung für den Bürger, im Mittelpunkt \citep[Vgl.][S. 27-28]{Schuppan2016}
\end{itemize}

Zusammenfassend gibt es neben dem Smart Government noch weitere Ansätze zur Verwaltungsmodernisierung in Deutschland, die ebenfalls zu einer effizienteren und bürgernahen Verwaltung beitragen können. 
Diese Ansätze umfassen unter anderem das Neue Steuerungsmodell und Verwaltung 4.0.
Dabei ist allerdings zu beachten, dass die verschiedenen Verwaltungsmodernisierungsarten teilweise inhaltlich übereinstimmen und nicht ganz klar abzugrenzen sind.
All diese Arten setzen mehr oder weniger auf Technologien um die Verwaltung zu Modernisieren.