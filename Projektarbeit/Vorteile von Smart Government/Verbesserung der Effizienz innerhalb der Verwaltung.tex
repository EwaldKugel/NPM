\subsection{Verbesserung der Effizienz innerhalb der Verwaltung}
Ein großes Kriterium bei der Betrachtung einer Verwaltungsmodernisierung ist die Effizienz, also die Wirtschaftlichkeit verschiedener Handlungen von Behörden.
Dabei können Smart Government-Lösungen die Effizienz der Verwaltung durch die Nutzung neuer Technologien und Innovationen erheblich verbessern.
Dies geschieht durch die Vereinfachung und Automatisierung einiger Abläufe bei der Einführung von Smart Government.
Weiterhin kann dies durch durch die Bereitstellung von Online-Diensten und Anwendungen erreicht werden, die es Bürgern und Unternehmen ermöglichen, ihre Arbeit online zu erledigen. 
Diese Technologien ermöglichen es den Verwaltungsstellen auch, automatische Warnungen und Benachrichtigungen an Bürger zu senden, die auf Abgabetermine oder Änderungen in den Verwaltungsprozessen hinweisen. 
Eine weitere Möglichkeit ist die Einführung von Chatbots, welche Bürger schneller und effizienter auf ihre Anfragen antworten, während die Verwaltungsstellen von den Einsparungen bei den Kosten für den Kundenservice profitieren. 
\par
Aber auch der Datenfluss zwischen den Verwaltungsstellen kann durch Smart Government beschleunigt werden. 
Zur sicheren und effizienten Datenübermittlung, -speicherung und -verarbeitung kann Cloud Computing und maschinelles Lernen von Behörden benutzt werden.
\par
Insgesamt kann Smart Government die Effizienz der Verwaltung durch die Automatisierung von Prozessen, die Verbesserung der Kommunikation und die Verringerung der Kosten für den Kundenservice erheblich verbessern. 
