\subsection{Erhöhung der Transparenz unter Smart Government}
Ein weiteres Qualitätsmerkmal und Kriterium bei der Verbesserung der Verwaltung bei der Verwaltungmodernisierung ist die Transparenz.
Dabei sollten besonders Verwaltungsakte möglichst transparent und leicht zugänglich gestaltet werden.
Smart Government kann die Transparenz der Verwaltung signifikant verbessern. 
Mit Smart Government können Regierungsstellen und Verwaltungen digitale Services und Anwendungen aufbauen, die eine schnelle, effiziente und transparente Kommunikation und Zusammenarbeit zwischen den Bürgern und der Verwaltung ermöglichen. 
\par
Das ermöglicht es der Regierung, ein einheitliches Portal zu erstellen, über das die Bürger auf öffentliche Dokumente und Informationen zugreifen können. 
Dadurch können Bürger leicht überprüfen, was in der Verwaltung passiert, und sich informieren, welche Richtlinien und Regeln in der Verwaltung gelten. 
Dies hilft den Bürgern, die Entscheidungsprozesse der Regierung besser zu verstehen und stärkt das Vertrauen in die Regierung.
\par
Weiterhin hilft Smart Government auch dabei, die Einhaltung von Richtlinien und Verfahren zu überwachen, indem es Unternehmen und Bürgern ermöglicht, sicher und schnell mit der Regierung zu kommunizieren und Dokumente und Anträge einzureichen. 
Dies verbessert die Transparenz der Verwaltung, indem es den Bürgern ermöglicht, den Status ihrer Anträge und Dokumente online zu verfolgen.
\par
Eine weitere Möglichkeit ist, öffentliche Datenbanken bereitzustellen, über die Bürger Zugang zu Informationen über verschiedene Verwaltungsprozesse, Ressourcen und Dienstleistungen erhalten können. (OPEN DATA)
Dies ermöglicht es Bürgern, über verschiedene Regierungsprogramme und Initiativen zu lernen und erhöht die Transparenz der Verwaltung.
\par
In Kombination können diese Technologien der Regierung helfen, eine benutzerfreundliche, effiziente und transparente Verwaltung zu schaffen, die den Bürgern hilft, besser mit der Regierung und ihren Dienstleistungen zu interagieren. 
Dies wird die Transparenz der Verwaltung und das Vertrauen der Bürger in sie erhöhen.