\section{Chancen und Risiken von Smart Government}
Die Verwaltungsmodernisierung hat das Ziel, die Effizienz, Wirksamkeit und Bürgernähe der Verwaltung zu verbessern.
Ein Ansatz zur Verwaltungsmodernisierung ist das Smart Government, das die Verwaltung durch den Einsatz von Informationstechnologie und digitalen Lösungen verbessern soll.
Beim Smart Government bewegen sich viele Anwendungsfelder in der öffentlichen Verwaltung auf einem schmalen Grat zwischen den Chancen, die Smart Government mitbringt und Risiken, wie Datenschutz, Überwachung und Fremdsteuerung \citep[Vgl.][]{Lucke2018}.
Von Lucke beschreibt vor allem den ethischen Aspekt bei der Nutzung von neuen Technologien, so sollte Smart Government nach den eigenen ethischen Vorstellungen zusammen mit der Bevölkerung gestaltet werden \citep[Vgl.][S.108]{Lucke2018}.
\par
Dabei sollten die Handlungsfelder, in denen autonome Technologien im Rahmen des Smart Governments eingesetzt werden, stark überdacht werden.
So ist in den meisten Fällen davon abzuraten autonome Computersysteme bei wichtigen Entscheidungen, zum Beispiel der Entscheidung über Haftstrafen, zu benutzen.
Laut Tonn und Stiefel könnte es sogar einen Verlust an Menschlichkeit geben, sollten solche Entscheidungen nicht mehr durch den Menschen getroffen werden \citep[Vgl.][S.322]{Tonn2012}.
\par
\subsection{Verbesserung der Effizienz innerhalb der Verwaltung}
\subsection{Erhöhung der Transparenz unter Smart Government}
Ein weiteres Qualitätsmerkmal und Kriterium bei der Verbesserung der Verwaltung bei der Verwaltungmodernisierung ist die Transparenz.
Dabei sollten besonders Verwaltungsakte möglichst transparent und leicht zugänglich gestaltet werden.
Smart Government kann die Transparenz der Verwaltung signifikant verbessern. 
Mit Smart Government können Regierungsstellen und Verwaltungen digitale Services und Anwendungen aufbauen, die eine schnelle, effiziente und transparente Kommunikation und Zusammenarbeit zwischen den Bürgern und der Verwaltung ermöglichen. 
\par
Das ermöglicht es der Regierung, ein einheitliches Portal zu erstellen, über das die Bürger auf öffentliche Dokumente und Informationen zugreifen können. 
Dadurch können Bürger leicht überprüfen, was in der Verwaltung passiert, und sich informieren, welche Richtlinien und Regeln in der Verwaltung gelten. 
Dies hilft den Bürgern, die Entscheidungsprozesse der Regierung besser zu verstehen und stärkt das Vertrauen in die Regierung.
\par
Weiterhin hilft Smart Government auch dabei, die Einhaltung von Richtlinien und Verfahren zu überwachen, indem es Unternehmen und Bürgern ermöglicht, sicher und schnell mit der Regierung zu kommunizieren und Dokumente und Anträge einzureichen. 
Dies verbessert die Transparenz der Verwaltung, indem es den Bürgern ermöglicht, den Status ihrer Anträge und Dokumente online zu verfolgen.
\par
Eine weitere Möglichkeit ist, öffentliche Datenbanken bereitzustellen, über die Bürger Zugang zu Informationen über verschiedene Verwaltungsprozesse, Ressourcen und Dienstleistungen erhalten können. (OPEN DATA)
Dies ermöglicht es Bürgern, über verschiedene Regierungsprogramme und Initiativen zu lernen und erhöht die Transparenz der Verwaltung.
\par
In Kombination können diese Technologien der Regierung helfen, eine benutzerfreundliche, effiziente und transparente Verwaltung zu schaffen, die den Bürgern hilft, besser mit der Regierung und ihren Dienstleistungen zu interagieren. 
Dies wird die Transparenz der Verwaltung und das Vertrauen der Bürger in sie erhöhen.
\subsection{Verstärkung der Bürgerbeteiligung an der Verwaltung} 