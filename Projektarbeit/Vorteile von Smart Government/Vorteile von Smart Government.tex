\section{Chancen und Risiken von Smart Government}
Die Verwaltungsmodernisierung hat das Ziel, die Effizienz, Wirksamkeit und Bürgernähe der Verwaltung zu verbessern.
Ein Ansatz zur Verwaltungsmodernisierung ist das Smart Government, das die Verwaltung durch den Einsatz von Informationstechnologie und digitalen Lösungen verbessern soll.
Beim Smart Government bewegen sich viele Anwendungsfelder in der öffentlichen Verwaltung auf einem schmalen Grat zwischen den Chancen, die Smart Government mitbringt und Risiken, wie Datenschutz, Überwachung und Fremdsteuerung \citep[Vgl.][]{Lucke2018}.
Von Lucke beschreibt vor allem den ethischen Aspekt bei der Nutzung von neuen Technologien, so sollte Smart Government nach den eigenen ethischen Vorstellungen zusammen mit der Bevölkerung gestaltet werden \citep[Vgl.][S.108]{Lucke2018}.
\par
Dabei sollten die Handlungsfelder, in denen autonome Technologien im Rahmen des Smart Governments eingesetzt werden, stark überdacht werden.
So ist in den meisten Fällen davon abzuraten autonome Computersysteme bei wichtigen Entscheidungen, zum Beispiel der Entscheidung über Haftstrafen, zu benutzen.
Laut Tonn und Stiefel könnte es sogar einen Verlust an Menschlichkeit geben, sollten solche Entscheidungen nicht mehr durch den Menschen getroffen werden \citep[Vgl.][S.322]{Tonn2012}.
\par
\subsection{Verbesserung der Effizienz innerhalb der Verwaltung}
Ein großes Kriterium bei der Betrachtung einer Verwaltungsmodernisierung ist die Effizienz, also die Wirtschaftlichkeit verschiedener Handlungen von Behörden.
Dabei können Smart Government-Lösungen die Effizienz der Verwaltung durch die Nutzung neuer Technologien und Innovationen erheblich verbessern.
Dies geschieht durch die Vereinfachung und Automatisierung einiger Abläufe bei der Einführung von Smart Government.
Weiterhin kann dies durch durch die Bereitstellung von Online-Diensten und Anwendungen erreicht werden, die es Bürgern und Unternehmen ermöglichen, ihre Arbeit online zu erledigen. 
Diese Technologien ermöglichen es den Verwaltungsstellen auch, automatische Warnungen und Benachrichtigungen an Bürger zu senden, die auf Abgabetermine oder Änderungen in den Verwaltungsprozessen hinweisen. 
Eine weitere Möglichkeit ist die Einführung von Chatbots, welche Bürger schneller und effizienter auf ihre Anfragen antworten, während die Verwaltungsstellen von den Einsparungen bei den Kosten für den Kundenservice profitieren. 
\par
Aber auch der Datenfluss zwischen den Verwaltungsstellen kann durch Smart Government beschleunigt werden. 
Zur sicheren und effizienten Datenübermittlung, -speicherung und -verarbeitung kann Cloud Computing und maschinelles Lernen von Behörden benutzt werden.
\par
Insgesamt kann Smart Government die Effizienz der Verwaltung durch die Automatisierung von Prozessen, die Verbesserung der Kommunikation und die Verringerung der Kosten für den Kundenservice erheblich verbessern. 

\input{Vorteile von Smart Government/Erhöhung der Transparenz unter Smart Government}
\subsection{Verstärkung der Bürgerbeteiligung an der Verwaltung}
Bürgerbeteiligung ist ein wesentlicher Bestandteil einer funktionierenden Demokratie. 
Sie ermöglicht es den Bürgern, Einfluss auf die Entscheidungen der Regierung zu nehmen, indem sie ihre Meinung kundtun und an der Entscheidungsfindung teilnehmen.
Gleichzeitig wird die Relevanz der Verwaltung erhöht, da es den Bürgern ermöglicht, ihre spezifischen Bedürfnisse anzusprechen und sie in die Entscheidungsfindung einzubeziehen.
Laut Schuppan wird die Qualität von Verwaltungen durch verbesserte Leistungsangebote, etwa durch digitalisierte Verwaltungsprozesse, die sich stärker mit dem Bürger beschäftigen, verbessert \citep[Vgl.][S.35]{Schuppan2016}.
Damit ist eine hohe Bürgerbeteiligung ein hohes Qualitätskriterium bei der Gestaltung einer Verwaltungsmodernisierung.
Folgend werden einige Projekte und Ansätze zur Erhöhung der Bürgerbeteiligung im Zuge des Smart Governments diskutiert.
\par
Generell leitet sich die Bürgerbeteiligung an der Verwaltung per Onlineangeboten an den Grundprinzipien des Open Government, also der Öffnung von Verwaltungsprozessen durch Behörden.
Eine wichtige Säule dieser Open Government Bewegung ist die Partizipation von Bürgern an der Verwaltung oder Politik (siehe Abbildung \ref{fig:DreiSäulenOpenGovernment}).
\begin{figure}[h]
 \centering
 \includegraphics{Assets/DreiSäulenOpenGovernment.png}
 \caption{Säulen des Open Governments \citep[][]{Leitner2018}.}
 \label{fig:DreiSäulenOpenGovernment}
\end{figure}
Beispiele für diese E-Partizipation an der Regierung sind digitale Volksabstimmungen oder Volksbefragungen.
Zwar gibt es Hürden wie die Identifikation der Bürger bei solchen E-Partizipationen, allerdings fällt die Abwegung von Chance durch Partizipation und das Risiko der Identifikation so aus, dass die Chance dem Ririsko überwiegt \citep[Vgl.][S.15]{Leitner2018}.
\par
Allgemein konzentriert sich Smart Government in Bezug auf die Bürgerbeteiligung darauf, Bürgern die Möglichkeit zu geben, sich aktiv an der Gestaltung ihrer Kommunen zu beteiligen. 
Es geht darum, die Fähigkeiten und Ideen der Bürger zu nutzen, um eine bessere Zusammenarbeit zwischen Bürgern und Verwaltung zu fördern.  
Eine Reihe von Initiativen und Projekten in Deutschland setzen sich für eine stärkere Bürgerbeteiligung ein, um mehr Mitbestimmung und Partizipation zu fördern \citep[Vgl.][S.18, 49-50]{Leitner2018}
\par 
Zusammengefasst lässt sich sagen, dass Smart Government den Bürgern die Chance bietet sich aktiv am politischen Prozess zu beteiligen und sich an der Entscheidungsfindung zu beteiligen. 
Dies kann durch neue Technologien umgesetzt und verbessert in der zukünftigen Verwaltung angewandt werden.
Dadurch wird die Transparenz in der Verwaltung gestärkt und die Bürger können ihre Interessen aktiv vertreten. 