\subsection{Verstärkung der Bürgerbeteiligung an der Verwaltung}
Bürgerbeteiligung ist ein wesentlicher Bestandteil einer funktionierenden Demokratie. 
Sie ermöglicht es den Bürgern, Einfluss auf die Entscheidungen der Regierung zu nehmen, indem sie ihre Meinung kundtun und an der Entscheidungsfindung teilnehmen.
Gleichzeitig wird die Relevanz der Verwaltung erhöht, da es den Bürgern ermöglicht, ihre spezifischen Bedürfnisse anzusprechen und sie in die Entscheidungsfindung einzubeziehen.
Damit ist eine hohe Bürgerbeteiligung ein hohes Qualitätskriterium bei der Gestaltung einer Verwaltungsmodernisierung.
Folgend werden einige Projekte und Ansätze zur Erhöhung der Bürgerbeteiligung im Zuge des Smart Governments diskutiert.
\par
In Deutschland gibt es verschiedene Initiativen, die die Bürgerbeteiligung an der Verwaltung mit Smart Government verbessern wollen. 
Ein Beispiel hierfür ist der Online-Dialog ``Deutschland debattiert". 
Online-Dialog "Deutschland debattiert" ist ein interaktives Online-Forum, das es Bürgern ermöglicht, sich an einer öffentlichen Debatte zu Themen zu beteiligen, die Deutschland betreffen. 
Diese Plattform bietet eine diskursive Methode, um eine breitere Beteiligung an der politischen Debatte zu ermöglichen. 
Ziel ist es, einen offenen Dialog zu ermöglichen, um die Meinungen und Ideen der Bürger zu einem bestimmten Thema zu sammeln und zu diskutieren. 
Ein Moderator leitet die Diskussion und die Teilnehmer können ihre Meinungen und Ideen in Form von Texten, Videos und Bildern miteinander teilen. 
Der Dialog kann dann auf verschiedenen sozialen Plattformen wie Facebook, Twitter und YouTube gefördert werden.
\par
Ein weiteres Projekt ist die Initiative ``Citizen Engagement''.
Citizen Engagement ist eine Initiative, die sich darauf konzentriert, Bürgern die Möglichkeit zu geben, sich aktiv an der Gestaltung ihrer Kommunen zu beteiligen. 
Es geht darum, die Fähigkeiten und Ideen der Bürger zu nutzen, um eine bessere Zusammenarbeit zwischen Bürgern und Verwaltung zu fördern.  
Eine Reihe von Initiativen und Projekten in Deutschland setzen sich für eine stärkere Bürgerbeteiligung ein, um mehr Mitbestimmung und Partizipation zu fördern.
\par
Eine andere Initiative zur Verbesserung der Bürgerbeteiligung ist ``Open Government''.
Das Ziel von Open Government ist, Transparenz und Beteiligung der Bürger an der Entscheidungsfindung zu erhöhen und die Öffentlichkeit über die Entscheidungsfindungsprozesse des Staates zu informieren.
Um dies zu erreichen, werden verschiedene Maßnahmen ergriffen, wie z.B. digitale Plattformen, die es Bürgern ermöglichen, sich an Entscheidungsprozessen zu beteiligen oder sich über die Ergebnisse zu informieren. 
Zudem werden offene Daten in Form von offenen Datenbanken, APIs und anderen digitalen Werkzeugen bereitgestellt, die es den Bürgern ermöglichen, Daten zu erheben und zu analysieren.
\par
Alle genannten Projekte tragen dazu bei, die Bürgerbeteiligung an der Verwaltung mit Smart Government zu verbessern. 
Sie ermöglichen den Bürgern, sich aktiv am politischen Prozess zu beteiligen und sich an der Entscheidungsfindung zu beteiligen. 
Dadurch wird die Transparenz in der Verwaltung gestärkt und die Bürger können ihre Interessen aktiv vertreten.