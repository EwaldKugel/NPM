\subsection{Kosten der Einführung von Smart Government}
% Der Kostenfaktor ist einer der größten Nachteile bei der Einführung intelligenter Regierung in Deutschland. 
% Der Einsatz von Technologie, um die Verwaltung zu verbessern, erfordert eine signifikante Investition in die notwendige Infrastruktur, die sich aufgrund der Komplexität und der erforderlichen technischen Fähigkeiten als sehr teuer erweisen kann. 
% Die Kosten für die Entwicklung der Infrastruktur und die Implementierung eines digitalen Systems, das alle staatlichen Dienste abdeckt, können zu erheblichen finanziellen Belastungen für die Regierung führen.
% \par
% Darüber hinaus können die Kosten ansteigen, wenn das System erweitert werden muss, um neue Technologien zu unterstützen oder die Sicherheit der Systeme zu verbessern. 
% Um die Kosten für die Implementierung intelligenter Regierungsmaßnahmen zu senken, muss die Bundesregierung Finanzhilfen von Unternehmen und Organisationen erhalten, die in die Entwicklung von Technologien und die Implementierung von Systemen investieren.
% \par
% Weiterhin müssen Investitionen in die Schulung von Personen getätigt werden, die das neue System verwalten und unterstützen, wodurch die Kosten weiter erhöht werden. 
% Auch die Kosten für den Betrieb des Systems müssen berücksichtigt werden, was zu weiteren erheblichen Kosten für die Regierung führen kann.
% \par
% Der Kostenfaktor ist ein ernstzunehmender Nachteil bei der Einführung von Smart Government. 
% Es ist notwendig, dass die Verwaltung Finanzhilfen von Organisationen erhält, um die Kosten der Implementierung des Systems zu senken und um die Kosten für die Schulung der Personen, die das System verwalten, zu decken.
% Allerdings gibt es durchaus auch Potentiale zur Kosteneinsparungen bei der Digitalisierung von papierbasierten Prozessen, wie zum Beispiel Einsparmöglichkeiten von Papierverbrauch, Energiekosten und notwendigen Eingriffen durch Verwaltungsmitarbeiter \citep[Vgl.][S.179]{von_Lucke_2016}.